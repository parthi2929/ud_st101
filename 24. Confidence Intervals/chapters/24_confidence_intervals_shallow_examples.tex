
% Default to the notebook output style

    


% Inherit from the specified cell style.




    
\documentclass[float=false,crop=false]{standalone}

    
    
\usepackage{../myipy2tex}  % NOTE WE ARE ASSSUMING THE STYLE FILE TO BE ONE FOLDER ABOVE

% if you need to cross reference to any raw tex file from this resultant tex file you  need to refer them here..
% it is not needed when you compile main.tex but make sure the labels are unique
\ifstandalone
\usepackage{xr-hyper} % Needed for external references
\externaldocument{"24. Confidence Intervals - Theory"} 
\fi




    


    


    \begin{document}
    
    
    \maketitle
    
    

    
    \section{Shallow Examples}\label{shallow-examples}

\subsection{\texorpdfstring{\(\sigma\) Known, Population Normal, Low
Sample
Size}{\textbackslash{}sigma Known, Population Normal, Low Sample Size}}\label{sigma-known-population-normal-low-sample-size}

    \emph{Let X equal the length of life of a 60-watt light bulb marketed by
a certain manufacturer. Assume that the distribution of X is
\(N(\mu, 1296)\). If a random sample of n = 27 bulbs is tested until
they burn out, yielding a sample mean of x = 1478 hours, find 95\%
confidence interval for \(\mu\)}.

\textbf{Solution:}
Here, its given that the population is Normal and also its population SD $\sigma$. So we could use equation \ref{eq:n4} right away. 
    Given\\
\(\sigma^2 = 1296 \therefore \sigma = 36\),\\
\(x = 1478\), \(1-\alpha = 0.95\),\\
\(z_{\frac{\alpha}{2}} = z_{0.025} = 1.96\), \(n = 27 \geq 5\)

Though sample size is \(< 30\), the population distribution is given as
normal already. Thus, our sampling distribution would still be a normal
distribution as below with 95\% confidence interval area.
% remove input part of cells with tag to_remove
    %((- if cell.metadata.hide_input -))
    \begin{Verbatim}[commandchars=\\\{\},fontsize=\footnotesize]
The tikzmagic extension is already loaded. To reload it, use:
  \%reload\_ext tikzmagic

    \end{Verbatim}
% remove input part of cells with tag to_remove
    %((- if cell.metadata.hide_input -))% remove input part of cells with tag to_remove
    %((- if cell.metadata.hide_input -))
    \begin{center}
    \adjustimage{max size={0.9\linewidth}{0.9\paperheight},min size={0.5\linewidth}{!}}{24_confidence_intervals_shallow_examples_files/24_confidence_intervals_shallow_examples_6_0.png}
    \end{center}
    { \hspace*{\fill} \\}
    
    We already know, in this sampling distribution, the mean
\(\overline{X} \to \mu\) and SD \(S \to \dfrac{\sigma}{\sqrt{n}}\). Thus
as we have already derived earlier,

\[
            \begin{aligned}
            Pr\Big( {\color{blue}{\overline{X}}} - 1.96S \leq x_0 \leq {\color{blue}{\overline{X}}} + 1.96S\Big) = 1-\alpha \nonumber \\
            Pr\Big( x_0 - 1.96S \leq {\color{blue}{\overline{X}}} \leq x_0 + 1.96S\Big) = 1-\alpha \nonumber \\            
            Pr\Big( x_0 - 1.96\frac{\sigma}{\sqrt{n}} \leq {\color{blue}{\mu}} \leq x_0 + 1.96\frac{\sigma}{\sqrt{n}}\Big) = 1-\alpha \nonumber \\
            \implies Pr\Big( 1478 - 1.96\frac{36}{\sqrt{27}} \leq \mu \leq 1478 + 1.96\frac{36}{\sqrt{27}}\Big) = 0.95 \nonumber \\
            Pr\Big( 1478 - 13.58 \leq \mu \leq 1478 + 13.58\Big) = 0.95 \nonumber \\
            Pr\Big( 1464.42 \leq \mu \leq 1491.58\Big) = 0.95 \nonumber \\
            \end{aligned}
\]

Thus the 95\% CI intervals are \([1464.42,1491.58]\). This does not
mean, \(\mu\) is inside this interval 95\% of the time. But simply, if
we are to take many such samples and their CIs, 95\% of those CIs would
contain \(\mu\). We do not know what those CIs would be because we do
not know the real \(\mu\).

    \subsection{\texorpdfstring{\(\sigma\) Known, Population not Normal,
High Sample
Size}{\textbackslash{}sigma Known, Population not Normal, High Sample Size}}\label{sigma-known-population-not-normal-high-sample-size}

    \emph{The operations manager of a large production plant would like to
estimate the mean amount of time a worker takes to assemble a new
electronic component. Assume that the standard deviation of this
assembly time is 3.6 minutes. After observing 120 workers assembling
similar devices, the manager noticed that their average time was 16.2
minutes. Construct a 92\% confidence interval for the mean assembly
time}.

    \textbf{\href{https://www.utdallas.edu/~mbaron/3341/Practice12.pdf}{Solution}:}

Given \(n = 120\) which is \(> 30\). The measurement in population is
\emph{mean amount of time} which is \emph{continuous}. Due to CLT, the
resulting sampling distribution of sample means from all sample sets of
size \(n=120\) would result in a normal continuous distribution. Since
population distribution is not normal (at least not given specifically),
we could expect our confidence interval to be \textbf{approximate} only.
Population SD \(\sigma\) is given as known which is 3.6 minutes. The
sample mean of sample set is 16.2 minutes, thus \(x = 16.2\)

Summarizing,

\(x = 16.2, n = 120, \sigma = 3.6\)\\
\(1 - \alpha = 0.92, \alpha = 0.08, \dfrac{\alpha}{2} = 0.04\)

Since resulting sampling distribution is normal,we could use Z
distribution. Remember, we use right tailed Z table here.
Recall \ref{sc:generalized-ci}. 
    Using
\href{http://www2.stat.duke.edu/~rcs46/lectures_2015/14-bayes1/z-table.pdf}{this}
table, we get

\[
z_\frac{\alpha}{2} = z_{0.04} = 1.75
\]
Using \ref{eq:n5}, 
    \[
\begin{aligned}
            Pr\Big( x - z_{\frac{\alpha}{2}}\frac{\sigma}{\sqrt{n}} \leq \mu \leq x + z_{\frac{\alpha}{2}}\frac{\sigma}{\sqrt{n}}\Big) \approx 1-\alpha \nonumber \\
            Pr\Big( 16.2 - 1.75\frac{3.6}{\sqrt{120}} \leq \mu \leq 16.2 + 1.75\frac{3.6}{\sqrt{120}}\Big) \approx 0.92 \nonumber \\
            Pr\Big( 16.2 - 0.575 \leq \mu \leq 16.2 + 0.575\Big) \approx 0.92 \nonumber \\
            Pr\Big( 15.625 \leq \mu \leq 16.775\Big) \approx 0.92 \nonumber \\
\end{aligned}
\]

Thus the 92\% confidence intervals for given sample set is
{[}15.625,16.775{]}

    \subsection{\texorpdfstring{\(\sigma\) Unknown, Population Normal, Low
Sample
Size}{\textbackslash{}sigma Unknown, Population Normal, Low Sample Size}}\label{sigma-unknown-population-normal-low-sample-size}

    To assess the accuracy of a laboratory scale, a standard weight that is
known to weigh 1 gram is repeatedly weighed 4 times. The resulting
measurements (in grams) are: 0.95, 1.02, 1.01, 0.98. Assume that the
weighings by the scale when the true weight is 1 gram are normally
distributed with mean \(\mu\). Use these data to compute a 95\%
confidence interval for \(\mu\)

\textbf{\href{https://www.utdallas.edu/~mbaron/3341/Practice12.pdf}{Solution}:}

The population is given as normally distributed with \(\sigma\) unknown.
Due to low sample size \(n=4 < 30\), the resultant sampling distribution
would be of student's \emph{t} distribution, than normal, so we need to
use that.

    Parameters of the sample set:
\begin{InVerbatim}[commandchars=\\\{\},fontsize=\scriptsize]
{\color{incolor}In[{\color{incolor}22}]:} \PY{n}{x} \PY{o}{=} \PY{p}{[}\PY{l+m+mf}{0.95}\PY{p}{,} \PY{l+m+mf}{1.02}\PY{p}{,} \PY{l+m+mf}{1.01}\PY{p}{,} \PY{l+m+mf}{0.98}\PY{p}{]}
         
         \PY{k}{def} \PY{n+nf}{get\PYZus{}metrics}\PY{p}{(}\PY{n}{x}\PY{p}{)}\PY{p}{:}
             \PY{k+kn}{from} \PY{n+nn}{math} \PY{k}{import} \PY{n}{sqrt}
             \PY{n}{n} \PY{o}{=} \PY{n+nb}{len}\PY{p}{(}\PY{n}{x}\PY{p}{)} \PY{c+c1}{\PYZsh{} sample size}
             \PY{n}{x\PYZus{}bar} \PY{o}{=} \PY{n+nb}{sum}\PY{p}{(}\PY{n}{x}\PY{p}{)}\PY{o}{/}\PY{n}{n}  \PY{c+c1}{\PYZsh{} unbiased sample mean}
             \PY{n}{var} \PY{o}{=} \PY{n+nb}{sum}\PY{p}{(} \PY{p}{[}\PY{p}{(}\PY{n}{x\PYZus{}i} \PY{o}{\PYZhy{}} \PY{n}{x\PYZus{}bar}\PY{p}{)}\PY{o}{*}\PY{o}{*}\PY{l+m+mi}{2} \PY{k}{for} \PY{n}{x\PYZus{}i} \PY{o+ow}{in} \PY{n}{x}\PY{p}{]} \PY{p}{)}\PY{o}{/}\PY{p}{(}\PY{n}{n}\PY{o}{\PYZhy{}}\PY{l+m+mi}{1}\PY{p}{)}
             \PY{n}{s} \PY{o}{=} \PY{n+nb}{round}\PY{p}{(}\PY{n}{sqrt}\PY{p}{(}\PY{n}{var}\PY{p}{)}\PY{p}{,}\PY{l+m+mi}{3}\PY{p}{)} \PY{c+c1}{\PYZsh{} unbiased sample SD}
             \PY{k}{return} \PY{n}{n}\PY{p}{,} \PY{n}{x\PYZus{}bar}\PY{p}{,} \PY{n}{var}\PY{p}{,} \PY{n}{s}
         
         \PY{n}{n}\PY{p}{,}\PY{n}{x\PYZus{}bar}\PY{p}{,}\PY{n}{\PYZus{}}\PY{p}{,}\PY{n}{s} \PY{o}{=} \PY{n}{get\PYZus{}metrics}\PY{p}{(}\PY{n}{x}\PY{p}{)}
         \PY{n+nb}{print}\PY{p}{(}\PY{l+s+s1}{\PYZsq{}}\PY{l+s+s1}{n:}\PY{l+s+si}{\PYZob{}\PYZcb{}}\PY{l+s+s1}{  x\PYZus{}bar:}\PY{l+s+si}{\PYZob{}\PYZcb{}}\PY{l+s+s1}{  s:}\PY{l+s+si}{\PYZob{}\PYZcb{}}\PY{l+s+s1}{\PYZsq{}}\PY{o}{.}\PY{n}{format}\PY{p}{(}\PY{n}{n}\PY{p}{,}\PY{n}{x\PYZus{}bar}\PY{p}{,}\PY{n}{s}\PY{p}{)}\PY{p}{)}
\end{InVerbatim}
    \begin{Verbatim}[commandchars=\\\{\},fontsize=\footnotesize]
n:4  x\_bar:0.99  s:0.032

    \end{Verbatim}

    Summarizing,

\(n = 4, \ x = 0.99, \ s = 0.032, \ 1-\alpha = 0.95\)

\(t_{\frac{\alpha}{2},(n-1)} = t_{\frac{0.05}{2},3} = t_{0.025,3}\)

Using
\href{http://pages.stat.wisc.edu/~ifischer/Intro_Stat/Lecture_Notes/APPENDIX/A5._Statistical_Tables/A5.2_-_t-distribution.pdf}{right
tailed t table}, \(t_{0.025,3} = 3.182\)

If we continued taking sample sets of this size \(n=4\), we would end up
getting a sampling distribution that has student's t distribution as
below.
% remove input part of cells with tag to_remove
    %((- if cell.metadata.hide_input -))% remove input part of cells with tag to_remove
    %((- if cell.metadata.hide_input -))
    \begin{center}
    \adjustimage{max size={0.9\linewidth}{0.9\paperheight},min size={0.5\linewidth}{!}}{24_confidence_intervals_shallow_examples_files/24_confidence_intervals_shallow_examples_21_0.png}
    \end{center}
    { \hspace*{\fill} \\}
    Thus, using \ref{eq:n6}, 
    \[
\begin{aligned}
    Pr\Big( x - t_{\frac{\alpha}{2},(n-1)}\frac{s}{\sqrt{n}} \leq \mu \leq x + t_{\frac{\alpha}{2},(n-1)}\frac{s}{\sqrt{n}}\Big) = 1-\alpha \nonumber \\
    Pr\Big( 0.99 - t_{(0.025,3)}\frac{0.032}{\sqrt{4}} \leq \mu \leq 0.99 + t_{(0.025,3)}\frac{0.032}{\sqrt{4}}\Big) = 0.95 \nonumber \\
    Pr\Big( 0.99 - 3.182\frac{0.032}{\sqrt{4}} \leq \mu \leq 0.99 + 3.182\frac{0.032}{\sqrt{4}}\Big) = 0.95
\end{aligned} \nonumber \\
\]
\begin{InVerbatim}[commandchars=\\\{\},fontsize=\scriptsize]
{\color{incolor}In[{\color{incolor}25}]:} \PY{k}{def} \PY{n+nf}{get\PYZus{}CI}\PY{p}{(}\PY{n}{x\PYZus{}bar}\PY{p}{,} \PY{n}{zrt}\PY{p}{,} \PY{n}{s}\PY{p}{,} \PY{n}{n}\PY{p}{)}\PY{p}{:}
             \PY{k+kn}{from} \PY{n+nn}{math} \PY{k}{import} \PY{n}{sqrt}
             \PY{n}{m} \PY{o}{=} \PY{n}{zrt}\PY{o}{*}\PY{p}{(}\PY{n}{s}\PY{o}{/}\PY{p}{(}\PY{n}{sqrt}\PY{p}{(}\PY{n}{n}\PY{p}{)}\PY{p}{)}\PY{p}{)}
             \PY{k}{return} \PY{p}{[}\PY{n}{x\PYZus{}bar}\PY{o}{\PYZhy{}}\PY{n}{m}\PY{p}{,}\PY{n}{x\PYZus{}bar}\PY{o}{+}\PY{n}{m}\PY{p}{]}
         
         \PY{n}{t} \PY{o}{=} \PY{l+m+mf}{3.182}
         \PY{n+nb}{print}\PY{p}{(}\PY{n}{get\PYZus{}CI}\PY{p}{(}\PY{n}{x\PYZus{}bar}\PY{p}{,} \PY{n}{t}\PY{p}{,} \PY{n}{s}\PY{p}{,} \PY{n}{n}\PY{p}{)}\PY{p}{)}
\end{InVerbatim}
    \begin{Verbatim}[commandchars=\\\{\},fontsize=\footnotesize]
[0.939088, 1.040912]

    \end{Verbatim}

    \(\therefore\) the 95\% CI in our case are, \[
\begin{aligned}
Pr\Big( 0.94 \leq \mu \leq 1.04\Big) = 0.95 \nonumber \\
\end{aligned}
\]

    \subsection{\texorpdfstring{\(\sigma\) Unknown, Population not Normal,
High Sample
Size}{\textbackslash{}sigma Unknown, Population not Normal, High Sample Size}}\label{sigma-unknown-population-not-normal-high-sample-size}

    \emph{In order to ensure efficient usage of a server, it is necessary to
estimate the mean number of concurrent users. According to records, the
sample mean and sample standard deviation of number of concurrent users
at 100 randomly selected times is 37.7 and 9.2, respectively.Construct a
90\% confidence interval for the mean number of concurrent users}.

    \textbf{\href{https://www.utdallas.edu/~mbaron/3341/Practice12.pdf}{Solution}}

The measurement at hand is mean number of concurrent users. This is a
continuous random variable. Irrespective of population distribution, if
sample size is large enough, due to CLT, eventually the sampling
distribution formed will be normal. Here \(n=100 > 30\), so we would at
least approximately could get good enough CI with 90\% confidence level
as asked.

Summarizing,

\(n=100, \ x = 37.7, \ s = 9.2\)\\
\(1 - \alpha = 0.9, \ \alpha=0.1, \ \frac{\alpha}{2} = 0.05\)\\
This time, we shall use code to find the right tailed z area,..
\begin{InVerbatim}[commandchars=\\\{\},fontsize=\scriptsize]
{\color{incolor}In[{\color{incolor}26}]:} \PY{k}{def} \PY{n+nf}{get\PYZus{}z}\PY{p}{(}\PY{n}{cl}\PY{p}{)}\PY{p}{:}
             \PY{k+kn}{from} \PY{n+nn}{scipy} \PY{k}{import} \PY{n}{stats}
             \PY{n}{alpha} \PY{o}{=} \PY{n+nb}{round}\PY{p}{(}\PY{p}{(}\PY{l+m+mi}{1} \PY{o}{\PYZhy{}} \PY{n}{cl}\PY{p}{)}\PY{o}{/}\PY{l+m+mi}{2}\PY{p}{,}\PY{l+m+mi}{3}\PY{p}{)}
             \PY{k}{return} \PY{p}{(}\PY{o}{\PYZhy{}}\PY{l+m+mi}{1}\PY{p}{)}\PY{o}{*}\PY{p}{(}\PY{n+nb}{round}\PY{p}{(}\PY{n}{stats}\PY{o}{.}\PY{n}{norm}\PY{o}{.}\PY{n}{ppf}\PY{p}{(}\PY{n}{alpha}\PY{p}{)}\PY{p}{,}\PY{l+m+mi}{3}\PY{p}{)}\PY{p}{)}  \PY{c+c1}{\PYZsh{} right tailing..}
         
         \PY{n+nb}{print}\PY{p}{(}\PY{n}{get\PYZus{}z}\PY{p}{(}\PY{l+m+mf}{0.90}\PY{p}{)}\PY{p}{)}
\end{InVerbatim}
    \begin{Verbatim}[commandchars=\\\{\},fontsize=\footnotesize]
1.645

    \end{Verbatim}

    Thus, \(z_{0.05} = 1.645\)
Using \ref{eq:n7}, but also using approximation as we do not know population distribution,
    \[
\begin{aligned}
    Pr\Big( x - z_{\frac{\alpha}{2}}\frac{s}{\sqrt{n}} \leq \mu \leq x + z_{\frac{\alpha}{2}}\frac{s}{\sqrt{n}}\Big) \approx 1-\alpha \nonumber \\
    Pr\Big( 37.7 - z_{0.05}\frac{9.2}{\sqrt{100}} \leq \mu \leq 37.7 + z_{0.05}\frac{9.2}{\sqrt{100}}\Big) \approx 0.9 \nonumber \\
    Pr\Big( 37.7 - 1.645\frac{9.2}{\sqrt{100}} \leq \mu \leq 37.7 + 1.645\frac{9.2}{\sqrt{100}}\Big) \approx 0.9 
\end{aligned}
\]
\begin{InVerbatim}[commandchars=\\\{\},fontsize=\scriptsize]
{\color{incolor}In[{\color{incolor}27}]:} \PY{n}{x}\PY{p}{,} \PY{n}{z}\PY{p}{,} \PY{n}{s}\PY{p}{,} \PY{n}{n} \PY{o}{=} \PY{l+m+mf}{37.7}\PY{p}{,} \PY{l+m+mf}{1.645}\PY{p}{,} \PY{l+m+mf}{9.2}\PY{p}{,} \PY{l+m+mi}{100}
         \PY{n+nb}{print}\PY{p}{(}\PY{n}{get\PYZus{}CI}\PY{p}{(}\PY{n}{x}\PY{p}{,} \PY{n}{z}\PY{p}{,} \PY{n}{s}\PY{p}{,} \PY{n}{n}\PY{p}{)}\PY{p}{)}
\end{InVerbatim}
    \begin{Verbatim}[commandchars=\\\{\},fontsize=\footnotesize]
[36.186600000000006, 39.2134]

    \end{Verbatim}

    Thus the desired 90\% CI intervals are {[}36.2,39.2{]}

Note: Since the sample size is high, even if \(t\) distribution is used,
result would be almost same, because at such high sample sizes, \(t\)
distribution would be almost identical to \(z\) distribution.

    \subsection{Difference between two means, Welch's 't'
interval}\label{difference-between-two-means-welchs-t-interval}

    \emph{The species, the deinopis and menneus, coexist in eastern
Australia. The following summary statistics were obtained on the size,
in millimeters, of the prey of the two species. Calculate the 95\%
confidence interval for the difference in their means. }

\begin{longtable}[]{@{}ll@{}}
\toprule
Adult Dinopis & Adult Menneus\tabularnewline
\midrule
\endhead
n=10 & m=10\tabularnewline
\(\overline{x}=10.26 mm\) & \(\overline{y}=9.02 mm\)\tabularnewline
\(s^2_x = (2.51)^2\) & \(s^2_y = (1.90)^2\)\tabularnewline
\bottomrule
\end{longtable}

    \textbf{\href{https://onlinecourses.science.psu.edu/stat414/node/203/}{Solution}}

\textbf{Given:}\\
Let \(\overline{X} = N(\mu_{\overline{x}},\sigma_{\overline{x}}^2)\) be
the random variable of sampling distribution for Adult Dinopis. And so
is \(\overline{Y} = N(\mu_{\overline{y}},\sigma_{\overline{y}}^2)\) for
Adult Menneus. Then we are given one sample set data frame from each
species.

\(\overline{x}_{1} = 10.26 mm, \ \ s_{\overline{x}}=2.51\ mm, \ \ n = 10\)\\
\(\overline{y}_{1} = 9.02 mm, \ \ s_{\overline{y}}=1.90\ mm, \ \ m = 10\)\\
\(1-\alpha = 0.95, \alpha = 0.05, \frac{\alpha}{2} = 0.025\)

\textbf{Approach:}\\
Note the \(\sigma_{x},\sigma_{y}\) are unknown. Also both \(n,m\) are
small \(n < 30, m < 30\). It is totally not needed that \(n=m\), but in
this case we have that.
Recalling \ref{eq:n10} and \ref{eq:n11},
    \[
    \begin{aligned}
        Pr\Big(  (\overline{X} - \overline{Y}) - t_{(\frac{\alpha}{2},r)}s_w \leq (\mu_{\overline{x}} - \mu_{\overline{y}}) \leq    (\overline{X} - \overline{Y}) + t_{(\frac{\alpha}{2},r)}s_w\Big) \approx 1-\alpha  \nonumber        
    \end{aligned}   
\]

\[
\begin{aligned}
\overline{x}_{1} - \overline{y}_{1} = 10.26 - 9.02 \nonumber \\
s_w = \sqrt{ {\dfrac{s_{\overline{x}}^2}{n}}   +  {\dfrac{s_{\overline{y}}^2}{m}}} = 
\sqrt{ {\dfrac{{2.51}^2}{10}}   +  {\dfrac{{1.90}^2}{10}}  } \nonumber \\
r = \dfrac{  \Big(  \dfrac{s_x^2}{n} + \dfrac{s_y^2}{m}   \Big)^2  }{ \dfrac{1}{n-1}\Big(\dfrac{s_x^2}{n}\Big)^2 + \dfrac{1}{m-1}\Big(\dfrac{s_y^2}{m}\Big)^2  } = 
\dfrac{  \Big(  \dfrac{{2.51}^2}{10} + \dfrac{{1.90}^2}{10}   \Big)^2  }{ \dfrac{1}{9}\Big(\dfrac{{2.51}^2}{10}\Big)^2 + \dfrac{1}{9}\Big(\dfrac{{1.90}^2}{10}\Big)^2  }
\end{aligned}
\]
\begin{InVerbatim}[commandchars=\\\{\},fontsize=\scriptsize]
{\color{incolor}In[{\color{incolor}28}]:} \PY{n}{x\PYZus{}1}\PY{p}{,} \PY{n}{y\PYZus{}1}\PY{p}{,} \PY{n}{s\PYZus{}xbar}\PY{p}{,} \PY{n}{s\PYZus{}ybar}\PY{p}{,} \PY{n}{n}\PY{p}{,} \PY{n}{m} \PY{o}{=} \PY{l+m+mf}{10.26}\PY{p}{,} \PY{l+m+mf}{9.02}\PY{p}{,} \PY{l+m+mf}{2.51}\PY{p}{,} \PY{l+m+mf}{1.90}\PY{p}{,} \PY{l+m+mi}{10}\PY{p}{,} \PY{l+m+mi}{10}
         
         \PY{n}{w\PYZus{}1} \PY{o}{=} \PY{n+nb}{round}\PY{p}{(}\PY{n}{x\PYZus{}1} \PY{o}{\PYZhy{}} \PY{n}{y\PYZus{}1}\PY{p}{,}\PY{l+m+mi}{3}\PY{p}{)}
         
         \PY{k}{def} \PY{n+nf}{get\PYZus{}s\PYZus{}w}\PY{p}{(}\PY{n}{s\PYZus{}x}\PY{p}{,} \PY{n}{s\PYZus{}y}\PY{p}{,}\PY{n}{n}\PY{p}{,}\PY{n}{m}\PY{p}{)}\PY{p}{:}
             \PY{n}{v\PYZus{}x}\PY{p}{,} \PY{n}{v\PYZus{}y} \PY{o}{=} \PY{p}{(}\PY{n}{s\PYZus{}x}\PY{o}{*}\PY{o}{*}\PY{l+m+mi}{2}\PY{p}{)}\PY{o}{/}\PY{n}{n}\PY{p}{,} \PY{p}{(}\PY{n}{s\PYZus{}y}\PY{o}{*}\PY{o}{*}\PY{l+m+mi}{2}\PY{p}{)}\PY{o}{/}\PY{n}{m}
             \PY{k+kn}{from} \PY{n+nn}{math} \PY{k}{import} \PY{n}{sqrt}
             \PY{k}{return} \PY{n+nb}{round}\PY{p}{(}\PY{n}{sqrt}\PY{p}{(}\PY{n}{v\PYZus{}x} \PY{o}{+} \PY{n}{v\PYZus{}y}\PY{p}{)}\PY{p}{,}\PY{l+m+mi}{4}\PY{p}{)}
         
         \PY{n}{s\PYZus{}w} \PY{o}{=} \PY{n}{get\PYZus{}s\PYZus{}w}\PY{p}{(}\PY{n}{s\PYZus{}xbar}\PY{p}{,} \PY{n}{s\PYZus{}ybar}\PY{p}{,} \PY{n}{n}\PY{p}{,} \PY{n}{m}\PY{p}{)}
         
         \PY{k}{def} \PY{n+nf}{get\PYZus{}r}\PY{p}{(}\PY{n}{s\PYZus{}x}\PY{p}{,} \PY{n}{s\PYZus{}y}\PY{p}{,}\PY{n}{n}\PY{p}{,}\PY{n}{m}\PY{p}{)}\PY{p}{:}
             \PY{n}{v\PYZus{}x}\PY{p}{,} \PY{n}{v\PYZus{}y} \PY{o}{=} \PY{p}{(}\PY{n}{s\PYZus{}x}\PY{o}{*}\PY{o}{*}\PY{l+m+mi}{2}\PY{p}{)}\PY{o}{/}\PY{n}{n}\PY{p}{,} \PY{p}{(}\PY{n}{s\PYZus{}y}\PY{o}{*}\PY{o}{*}\PY{l+m+mi}{2}\PY{p}{)}\PY{o}{/}\PY{n}{m}
             \PY{n}{num} \PY{o}{=} \PY{p}{(}\PY{n}{v\PYZus{}x} \PY{o}{+} \PY{n}{v\PYZus{}y}\PY{p}{)}\PY{o}{*}\PY{o}{*}\PY{l+m+mi}{2}
             \PY{n}{den\PYZus{}1} \PY{o}{=} \PY{p}{(}\PY{l+m+mi}{1}\PY{o}{/}\PY{p}{(}\PY{n}{n}\PY{o}{\PYZhy{}}\PY{l+m+mi}{1}\PY{p}{)}\PY{p}{)}\PY{o}{*}\PY{p}{(}\PY{p}{(}\PY{n}{v\PYZus{}x}\PY{p}{)}\PY{o}{*}\PY{o}{*}\PY{l+m+mi}{2}\PY{p}{)}
             \PY{n}{den\PYZus{}2} \PY{o}{=} \PY{p}{(}\PY{l+m+mi}{1}\PY{o}{/}\PY{p}{(}\PY{n}{m}\PY{o}{\PYZhy{}}\PY{l+m+mi}{1}\PY{p}{)}\PY{p}{)}\PY{o}{*}\PY{p}{(}\PY{p}{(}\PY{n}{v\PYZus{}y}\PY{p}{)}\PY{o}{*}\PY{o}{*}\PY{l+m+mi}{2}\PY{p}{)}
             \PY{n}{r} \PY{o}{=} \PY{n}{num} \PY{o}{/} \PY{p}{(}\PY{n}{den\PYZus{}1} \PY{o}{+} \PY{n}{den\PYZus{}2}\PY{p}{)}
             \PY{k+kn}{from} \PY{n+nn}{math} \PY{k}{import} \PY{n}{modf}
             \PY{k}{return} \PY{n}{modf}\PY{p}{(}\PY{n}{r}\PY{p}{)}\PY{p}{[}\PY{l+m+mi}{1}\PY{p}{]}
         
         \PY{n}{r} \PY{o}{=} \PY{n}{get\PYZus{}r}\PY{p}{(}\PY{n}{s\PYZus{}xbar}\PY{p}{,} \PY{n}{s\PYZus{}ybar}\PY{p}{,} \PY{n}{n}\PY{p}{,} \PY{n}{m}\PY{p}{)}
         
         \PY{n+nb}{print}\PY{p}{(}\PY{l+s+s1}{\PYZsq{}}\PY{l+s+s1}{x\PYZus{}bar \PYZhy{} y\PYZus{}bar:}\PY{l+s+si}{\PYZob{}\PYZcb{}}\PY{l+s+s1}{, s\PYZus{}w:}\PY{l+s+si}{\PYZob{}\PYZcb{}}\PY{l+s+s1}{, r:}\PY{l+s+si}{\PYZob{}\PYZcb{}}\PY{l+s+s1}{\PYZsq{}}\PY{o}{.}\PY{n}{format}\PY{p}{(}\PY{n}{w\PYZus{}1}\PY{p}{,} \PY{n}{s\PYZus{}w}\PY{p}{,} \PY{n}{r}\PY{p}{)}\PY{p}{)}
         
         \PY{c+c1}{\PYZsh{} calculate t value}
         \PY{n}{cl} \PY{o}{=} \PY{l+m+mf}{0.95}
         \PY{n}{half\PYZus{}alpha} \PY{o}{=} \PY{n+nb}{round}\PY{p}{(}\PY{p}{(}\PY{l+m+mi}{1} \PY{o}{\PYZhy{}} \PY{n}{cl}\PY{p}{)}\PY{o}{/}\PY{l+m+mi}{2}\PY{p}{,}\PY{l+m+mi}{3}\PY{p}{)}
         \PY{k+kn}{from} \PY{n+nn}{scipy} \PY{k}{import} \PY{n}{stats}
         \PY{n}{t} \PY{o}{=} \PY{n+nb}{round}\PY{p}{(}\PY{n}{stats}\PY{o}{.}\PY{n}{t}\PY{o}{.}\PY{n}{ppf}\PY{p}{(}\PY{l+m+mi}{1}\PY{o}{\PYZhy{}}\PY{n}{half\PYZus{}alpha}\PY{p}{,} \PY{n}{r}\PY{p}{)}\PY{p}{,}\PY{l+m+mi}{3}\PY{p}{)}
         
         \PY{n+nb}{print}\PY{p}{(}\PY{l+s+s1}{\PYZsq{}}\PY{l+s+s1}{t:}\PY{l+s+s1}{\PYZsq{}} \PY{o}{+} \PY{n+nb}{str}\PY{p}{(}\PY{n}{t}\PY{p}{)}\PY{p}{)}
\end{InVerbatim}
    \begin{Verbatim}[commandchars=\\\{\},fontsize=\footnotesize]
x\_bar - y\_bar:1.24, s\_w:0.9955, r:16.0
t:2.12

    \end{Verbatim}

    \[
    \begin{aligned}
        Pr\Big(  (\overline{X} - \overline{Y}) - t_{(\frac{\alpha}{2},r)}s_w \leq (\mu_{\overline{x}} - \mu_{\overline{y}}) \leq    (\overline{X} - \overline{Y}) + t_{(\frac{\alpha}{2},r)}s_w\Big) \approx 1-\alpha  \nonumber \\    
        Pr\Big(  1.24 - (2.12)(0.9955) \leq (\mu_{\overline{x}} - \mu_{\overline{y}}) \leq      1.24 + (2.12)(0.9955)\Big) \approx 0.95  \nonumber \\    
    \end{aligned}   
\]
\begin{InVerbatim}[commandchars=\\\{\},fontsize=\scriptsize]
{\color{incolor}In[{\color{incolor}29}]:} \PY{n}{cilow}\PY{p}{,} \PY{n}{cihigh} \PY{o}{=} \PY{n+nb}{round}\PY{p}{(}\PY{p}{(}\PY{n}{w\PYZus{}1} \PY{o}{\PYZhy{}} \PY{n}{t}\PY{o}{*}\PY{n}{s\PYZus{}w}\PY{p}{)}\PY{p}{,}\PY{l+m+mi}{4}\PY{p}{)}\PY{p}{,}\PY{n+nb}{round}\PY{p}{(}\PY{p}{(}\PY{n}{w\PYZus{}1} \PY{o}{+} \PY{n}{t}\PY{o}{*}\PY{n}{s\PYZus{}w}\PY{p}{)}\PY{p}{,}\PY{l+m+mi}{4}\PY{p}{)}
         \PY{n+nb}{print}\PY{p}{(}\PY{n}{cilow}\PY{p}{,} \PY{n}{cihigh}\PY{p}{)}
\end{InVerbatim}
    \begin{Verbatim}[commandchars=\\\{\},fontsize=\footnotesize]
-0.8705 3.3505

    \end{Verbatim}

    \[
    \begin{aligned}
        Pr( -0.87 \leq (\mu_{\overline{x}} - \mu_{\overline{y}}) \leq   3.35) \approx 0.95  \nonumber \\    
    \end{aligned}   
\]

Thus the 95\% confidence intervals for the difference of sample means of
given problem is \$(-0.87, 3.35)

    \subsection{Difference between two
proportions}\label{difference-between-two-proportions}

\emph{Duncan is investigating if residents of a city support the
construction of a new high school. He's curious about the difference of
opinion between residents in the North and South parts of the city. He
obtained separate random samples of voters from each region. Here are
the results:}

\begin{longtable}[]{@{}lll@{}}
\toprule
Supports Construction? & North & South\tabularnewline
\midrule
\endhead
Yes & 54 & 77\tabularnewline
No & 66 & 63\tabularnewline
Total & 120 & 140\tabularnewline
\bottomrule
\end{longtable}

\emph{Duncan wants to use these results to construct a 90\% confidence
interval to estimate the difference in the proportion of residents in
these regions who support the construction project \((p_S-p_N)\). Assume
that all of the conditions for inference have been met. Calculate 90\%
confidence interval based on Duncan's samples}

    \textbf{\href{https://www.khanacademy.org/math/ap-statistics/two-sample-inference/two-sample-z-interval-proportions/v/calculating-two-sample-z-interval-confidence-interval-for-difference-of-proportions}{Solution}:}

Conveniently the sample sizes are high, so we could assume normal
approximations for sampling distributions of sample proportions for both
North and South parts of the city.

\textbf{Given:}\\
Let \(\dfrac{Y_S}{n_S} = N\Big(p_1, \dfrac{p_1q_1}{n_1}\Big)\) represent
sampling distribution for South. Similarly,
\(\dfrac{Y_N}{n_N} = N\Big(p_2, \dfrac{p_2q_2}{n_2}\Big)\) for North.

We have the test statistic as follows.\\
\(\hat{p_S} = \dfrac{y_S}{n_S} = \dfrac{77}{140}, \hat{q_S} = \dfrac{y_S}{n_S} = 1 - \dfrac{77}{140}\)\\
\(\hat{p_N} = \dfrac{y_N}{n_N} = \dfrac{54}{120}, \hat{q_N} = 1 - \dfrac{y_N}{n_N} = 1 - \dfrac{54}{120}\)\\
\(1 - \alpha = 0.90, \alpha = 0.1, \dfrac{\alpha}{2} = 0.05\)
\begin{InVerbatim}[commandchars=\\\{\},fontsize=\scriptsize]
{\color{incolor}In[{\color{incolor}12}]:} \PY{n}{t\PYZus{}s} \PY{o}{=} \PY{p}{[}\PY{l+m+mi}{77}\PY{o}{/}\PY{l+m+mi}{140}\PY{p}{,} \PY{l+m+mi}{1}\PY{o}{\PYZhy{}}\PY{p}{(}\PY{l+m+mi}{77}\PY{o}{/}\PY{l+m+mi}{140}\PY{p}{)}\PY{p}{,} \PY{l+m+mi}{54}\PY{o}{/}\PY{l+m+mi}{120}\PY{p}{,} \PY{l+m+mi}{1}\PY{o}{\PYZhy{}}\PY{p}{(}\PY{l+m+mi}{54}\PY{o}{/}\PY{l+m+mi}{120}\PY{p}{)}\PY{p}{]}
         \PY{n}{t\PYZus{}s} \PY{o}{=} \PY{p}{[}\PY{l+s+s1}{\PYZsq{}}\PY{l+s+si}{\PYZpc{}0.3f}\PY{l+s+s1}{\PYZsq{}} \PY{o}{\PYZpc{}} \PY{n}{e} \PY{k}{for} \PY{n}{e} \PY{o+ow}{in} \PY{n}{t\PYZus{}s}\PY{p}{]}
         \PY{n}{t\PYZus{}s} \PY{o}{=} \PY{p}{[}\PY{n+nb}{float}\PY{p}{(}\PY{n}{i}\PY{p}{)} \PY{k}{for} \PY{n}{i} \PY{o+ow}{in} \PY{n}{t\PYZus{}s}\PY{p}{]}
         \PY{p}{[}\PY{n}{p\PYZus{}s}\PY{p}{,} \PY{n}{q\PYZus{}s}\PY{p}{,} \PY{n}{p\PYZus{}n}\PY{p}{,} \PY{n}{q\PYZus{}n}\PY{p}{]} \PY{o}{=} \PY{n}{t\PYZus{}s}
         \PY{n+nb}{print}\PY{p}{(}\PY{n}{p\PYZus{}s}\PY{p}{,} \PY{n}{q\PYZus{}s}\PY{p}{,} \PY{n}{p\PYZus{}n}\PY{p}{,} \PY{n}{q\PYZus{}n}\PY{p}{)}
\end{InVerbatim}
    \begin{Verbatim}[commandchars=\\\{\},fontsize=\footnotesize]
0.55 0.45 0.45 0.55

    \end{Verbatim}

    \(\therefore \hat{p_S} = 0.55,\hat{q_S} = 0.45,\hat{p_N} = 0.45,\hat{p_N} = 0.55\)
Recalling \ref{eq:n13}, we need to find,
    \[
    \begin{aligned}
    Pr\Bigg(  -z_{\frac{\alpha}{2}} \leq \dfrac{(\hat{p_S} - \hat{p_N}) - (p_S - p_N) }{\sqrt{ {\frac{\hat{p_S}\hat{q_S}}{n_S}}   +  {\frac{\hat{p_N}{\hat{q_N}}}{n_N}}  }} \leq    z_{\frac{\alpha}{2}}\Bigg) \approx 1-\alpha = 0.90  \nonumber
    \end{aligned}  
\]
\begin{InVerbatim}[commandchars=\\\{\},fontsize=\scriptsize]
{\color{incolor}In[{\color{incolor}16}]:} \PY{n}{diff} \PY{o}{=} \PY{n+nb}{round}\PY{p}{(}\PY{n}{p\PYZus{}s} \PY{o}{\PYZhy{}} \PY{n}{p\PYZus{}n}\PY{p}{,}\PY{l+m+mi}{3}\PY{p}{)}
         
         \PY{n}{n\PYZus{}s}\PY{p}{,} \PY{n}{n\PYZus{}n} \PY{o}{=} \PY{l+m+mi}{140}\PY{p}{,}\PY{l+m+mi}{120}
         \PY{k+kn}{from} \PY{n+nn}{math} \PY{k}{import} \PY{n}{sqrt}
         \PY{n}{w\PYZus{}sd} \PY{o}{=} \PY{n+nb}{round}\PY{p}{(}\PY{n}{sqrt}\PY{p}{(}\PY{p}{(}\PY{n}{p\PYZus{}s}\PY{o}{*}\PY{n}{q\PYZus{}s}\PY{o}{/}\PY{n}{n\PYZus{}s}\PY{p}{)} \PY{o}{+} \PY{p}{(}\PY{n}{p\PYZus{}n}\PY{o}{*}\PY{n}{q\PYZus{}n}\PY{o}{/}\PY{n}{n\PYZus{}n}\PY{p}{)}\PY{p}{)}\PY{p}{,}\PY{l+m+mi}{3}\PY{p}{)}
         
         
         \PY{c+c1}{\PYZsh{} get Z}
         \PY{n}{cl} \PY{o}{=} \PY{l+m+mf}{0.90}
         \PY{k+kn}{from} \PY{n+nn}{scipy} \PY{k}{import} \PY{n}{stats}
         \PY{n}{alpha} \PY{o}{=} \PY{l+m+mi}{1} \PY{o}{\PYZhy{}} \PY{n}{cl}
         \PY{n}{z} \PY{o}{=} \PY{p}{(}\PY{o}{\PYZhy{}}\PY{l+m+mi}{1}\PY{p}{)}\PY{o}{*}\PY{n+nb}{round}\PY{p}{(}\PY{n}{stats}\PY{o}{.}\PY{n}{norm}\PY{o}{.}\PY{n}{ppf}\PY{p}{(}\PY{n}{alpha}\PY{o}{/}\PY{l+m+mi}{2}\PY{p}{)}\PY{p}{,}\PY{l+m+mi}{3}\PY{p}{)}
         
         \PY{n+nb}{print}\PY{p}{(}\PY{n}{diff}\PY{p}{,} \PY{n}{w\PYZus{}sd}\PY{p}{,} \PY{n}{z}\PY{p}{)}
\end{InVerbatim}
    \begin{Verbatim}[commandchars=\\\{\},fontsize=\footnotesize]
0.1 0.062 1.645

    \end{Verbatim}

    Substituting, we get,

\[
    \begin{aligned}
    Pr\Big(  -1.645 \leq \dfrac{0.1 - (p_S - p_N) }{0.062} \leq  1.645 \Big) \approx 0.90  \nonumber \\
    Pr\Big(  (-1.645)0.062 \leq 0.1 - (p_S - p_N) \leq  (1.645)0.062 \Big) \approx 0.90 \nonumber \\
    Pr\Big(  0.1 - (1.645)0.062 \leq (p_S - p_N) \leq  0.1 + (1.645)0.062 \Big) \approx 0.90 \nonumber \\
    \end{aligned}  
\]
\begin{InVerbatim}[commandchars=\\\{\},fontsize=\scriptsize]
{\color{incolor}In[{\color{incolor}18}]:} \PY{n}{cilow}\PY{p}{,} \PY{n}{cihigh} \PY{o}{=} \PY{n+nb}{round}\PY{p}{(}\PY{n}{diff} \PY{o}{\PYZhy{}} \PY{n}{z}\PY{o}{*}\PY{n}{w\PYZus{}sd}\PY{p}{,}\PY{l+m+mi}{3}\PY{p}{)}\PY{p}{,} \PY{n+nb}{round}\PY{p}{(}\PY{n}{diff} \PY{o}{+} \PY{n}{z}\PY{o}{*}\PY{n}{w\PYZus{}sd}\PY{p}{,}\PY{l+m+mi}{3}\PY{p}{)}
         \PY{n+nb}{print}\PY{p}{(}\PY{n}{cilow}\PY{p}{,} \PY{n}{cihigh}\PY{p}{)}
\end{InVerbatim}
    \begin{Verbatim}[commandchars=\\\{\},fontsize=\footnotesize]
-0.002 0.202

    \end{Verbatim}

    Thus the 90\% CI intervals for the difference between proportions are
\((-0.002, 0.202)\). That is,

\[
\begin{aligned}
Pr\Big(  -0.002 \leq (p_S - p_N) \leq  0.202 \Big) \approx 0.90 \nonumber 
\end{aligned}
\]

    \section{Useful Snippets}\label{useful-snippets}

\subsection{Python}\label{python}

\textbf{Get t score}

Could be useful, when you have significance level \(\alpha\) and degrees
of freedom \(df = n-1\), and have to calculate corresponding t score
\begin{InVerbatim}[commandchars=\\\{\},fontsize=\scriptsize]
{\color{incolor}In[{\color{incolor}30}]:} \PY{k}{def} \PY{n+nf}{get\PYZus{}t}\PY{p}{(}\PY{n}{cl}\PY{p}{,} \PY{n}{n}\PY{p}{)}\PY{p}{:}
             \PY{k+kn}{from} \PY{n+nn}{scipy} \PY{k}{import} \PY{n}{stats}
             \PY{n}{half\PYZus{}alpha} \PY{o}{=} \PY{n+nb}{round}\PY{p}{(}\PY{p}{(}\PY{l+m+mi}{1} \PY{o}{\PYZhy{}} \PY{n}{cl}\PY{p}{)}\PY{o}{/}\PY{l+m+mi}{2}\PY{p}{,}\PY{l+m+mi}{3}\PY{p}{)}
             \PY{k}{return} \PY{n+nb}{round}\PY{p}{(}\PY{n}{stats}\PY{o}{.}\PY{n}{t}\PY{o}{.}\PY{n}{ppf}\PY{p}{(}\PY{l+m+mi}{1}\PY{o}{\PYZhy{}}\PY{n}{half\PYZus{}alpha}\PY{p}{,} \PY{n}{n}\PY{o}{\PYZhy{}}\PY{l+m+mi}{1}\PY{p}{)}\PY{p}{,}\PY{l+m+mi}{3}\PY{p}{)}
         
         \PY{n}{cl} \PY{o}{=} \PY{l+m+mf}{0.95}  \PY{c+c1}{\PYZsh{} confidence level}
         \PY{n}{n} \PY{o}{=} \PY{l+m+mi}{4}      \PY{c+c1}{\PYZsh{} sample size}
         \PY{n+nb}{print}\PY{p}{(}\PY{n}{get\PYZus{}t}\PY{p}{(}\PY{n}{cl}\PY{p}{,} \PY{n}{n}\PY{p}{)}\PY{p}{)}
\end{InVerbatim}
    \begin{Verbatim}[commandchars=\\\{\},fontsize=\footnotesize]
3.182

    \end{Verbatim}

    \textbf{Get Z score}

Could be useful, when you have significance level \(\alpha\) and have to
calculate corresponding Z score. Remember to always check if you need
left tailed area or right tailed.
\begin{InVerbatim}[commandchars=\\\{\},fontsize=\scriptsize]
{\color{incolor}In[{\color{incolor}31}]:} \PY{k}{def} \PY{n+nf}{get\PYZus{}z}\PY{p}{(}\PY{n}{cl}\PY{p}{)}\PY{p}{:}
             \PY{c+c1}{\PYZsh{}NOTE:returns right tailed area as that is mostly used in CI}
             \PY{k+kn}{from} \PY{n+nn}{scipy} \PY{k}{import} \PY{n}{stats}
             \PY{n}{alpha} \PY{o}{=} \PY{n+nb}{round}\PY{p}{(}\PY{p}{(}\PY{l+m+mi}{1} \PY{o}{\PYZhy{}} \PY{n}{cl}\PY{p}{)}\PY{o}{/}\PY{l+m+mi}{2}\PY{p}{,}\PY{l+m+mi}{3}\PY{p}{)}
             \PY{k}{return} \PY{p}{(}\PY{o}{\PYZhy{}}\PY{l+m+mi}{1}\PY{p}{)}\PY{o}{*}\PY{n+nb}{round}\PY{p}{(}\PY{n}{stats}\PY{o}{.}\PY{n}{norm}\PY{o}{.}\PY{n}{ppf}\PY{p}{(}\PY{n}{alpha}\PY{p}{)}\PY{p}{,}\PY{l+m+mi}{3}\PY{p}{)}  \PY{c+c1}{\PYZsh{} right tailing..}
         
         \PY{n}{cl} \PY{o}{=} \PY{l+m+mf}{0.95}
         \PY{n+nb}{print}\PY{p}{(}\PY{n}{get\PYZus{}z}\PY{p}{(}\PY{n}{cl}\PY{p}{)}\PY{p}{)}
\end{InVerbatim}
    \begin{Verbatim}[commandchars=\\\{\},fontsize=\footnotesize]
1.96

    \end{Verbatim}

    \textbf{Z and T distribution}

Plotting a z and t distribution.
\begin{InVerbatim}[commandchars=\\\{\},fontsize=\scriptsize]
{\color{incolor}In[{\color{incolor}32}]:} \PY{o}{\PYZpc{}}\PY{k}{matplotlib} inline
         \PY{k+kn}{from} \PY{n+nn}{scipy}\PY{n+nn}{.}\PY{n+nn}{stats} \PY{k}{import} \PY{n}{t}\PY{p}{,} \PY{n}{norm}
         \PY{k+kn}{import} \PY{n+nn}{numpy} \PY{k}{as} \PY{n+nn}{np}
         \PY{k+kn}{import} \PY{n+nn}{matplotlib}\PY{n+nn}{.}\PY{n+nn}{pyplot} \PY{k}{as} \PY{n+nn}{plt}
         
         \PY{n}{n} \PY{o}{=} \PY{l+m+mi}{3}
         \PY{n}{df} \PY{o}{=} \PY{n}{n}\PY{o}{\PYZhy{}}\PY{l+m+mi}{1}
         \PY{n}{fig}\PY{p}{,}\PY{n}{ax} \PY{o}{=} \PY{n}{plt}\PY{o}{.}\PY{n}{subplots}\PY{p}{(}\PY{l+m+mi}{1}\PY{p}{,}\PY{l+m+mi}{1}\PY{p}{)}
         \PY{n}{x} \PY{o}{=} \PY{n}{np}\PY{o}{.}\PY{n}{linspace}\PY{p}{(}\PY{n}{t}\PY{o}{.}\PY{n}{ppf}\PY{p}{(}\PY{l+m+mf}{0.01}\PY{p}{,}\PY{n}{df}\PY{p}{)}\PY{p}{,} \PY{n}{t}\PY{o}{.}\PY{n}{ppf}\PY{p}{(}\PY{l+m+mf}{0.99}\PY{p}{,}\PY{n}{df}\PY{p}{)}\PY{p}{,}\PY{l+m+mi}{100}\PY{p}{)}
         \PY{n}{ax}\PY{o}{.}\PY{n}{plot}\PY{p}{(}\PY{n}{x}\PY{p}{,} \PY{n}{t}\PY{o}{.}\PY{n}{pdf}\PY{p}{(}\PY{n}{x}\PY{p}{,}\PY{n}{df}\PY{p}{)}\PY{p}{,} \PY{n}{color}\PY{o}{=}\PY{l+s+s1}{\PYZsq{}}\PY{l+s+s1}{C0}\PY{l+s+s1}{\PYZsq{}}\PY{p}{)}  \PY{c+c1}{\PYZsh{} blue is t distribution}
         \PY{n}{ax}\PY{o}{.}\PY{n}{plot}\PY{p}{(}\PY{n}{x}\PY{p}{,} \PY{n}{norm}\PY{o}{.}\PY{n}{pdf}\PY{p}{(}\PY{n}{x}\PY{p}{)}\PY{p}{,} \PY{n}{color}\PY{o}{=}\PY{l+s+s1}{\PYZsq{}}\PY{l+s+s1}{C1}\PY{l+s+s1}{\PYZsq{}}\PY{p}{)}  \PY{c+c1}{\PYZsh{} red}
         \PY{n}{plt}\PY{o}{.}\PY{n}{show}\PY{p}{(}\PY{p}{)}
\end{InVerbatim}
    \begin{center}
    \adjustimage{max size={0.9\linewidth}{0.9\paperheight},min size={0.5\linewidth}{!}}{24_confidence_intervals_shallow_examples_files/24_confidence_intervals_shallow_examples_59_0.png}
    \end{center}
    { \hspace*{\fill} \\}
    
    \subsection{Tikz in Ipython}\label{tikz-in-ipython}

Some parts of this book including this section are created using ipython
notebooks and thus few figures which needed to be constructed via tikz
needed an extension. Below figures are created via tikz by using an
ipython extension called
\href{https://github.com/mkrphys/ipython-tikzmagic}{tikzmagic}, so the
format is slightly different for preamble. However, for tikz users, the
essence could be easily captured.

For first time usage (or after reset and clear of notebook), always load
tikz as below.

\begin{verbatim}
%load_ext tikzmagic
\end{verbatim}

Also note, preamble is placed in a separate code cell above, because
ipython needs magic commands to start as first line in cells. Here, tikz
execution needs a magic command in subsequent cell.

\textbf{Z distribution:}
\begin{InVerbatim}[commandchars=\\\{\},fontsize=\scriptsize]
{\color{incolor}In[{\color{incolor}33}]:} \PY{n}{preamble} \PY{o}{=} \PY{l+s+s1}{\PYZsq{}\PYZsq{}\PYZsq{}}
         \PY{l+s+s1}{    }\PY{l+s+s1}{\PYZbs{}}\PY{l+s+s1}{pgfmathdeclarefunction}\PY{l+s+si}{\PYZob{}gauss\PYZcb{}}\PY{l+s+si}{\PYZob{}3\PYZcb{}}\PY{l+s+s1}{\PYZob{}}\PY{l+s+s1}{\PYZpc{}}
         \PY{l+s+s1}{      }\PY{l+s+s1}{\PYZbs{}}\PY{l+s+s1}{pgfmathparse}\PY{l+s+s1}{\PYZob{}}\PY{l+s+s1}{1/(\PYZsh{}3*sqrt(2*pi))*exp(\PYZhy{}((\PYZsh{}1\PYZhy{}\PYZsh{}2)\PYZca{}2)/(2*\PYZsh{}3\PYZca{}2))\PYZcb{}}\PY{l+s+s1}{\PYZpc{}}
         \PY{l+s+s1}{    \PYZcb{}}
         \PY{l+s+s1}{\PYZsq{}\PYZsq{}\PYZsq{}}
\end{InVerbatim}
    
\begin{InVerbatim}[commandchars=\\\{\},fontsize=\scriptsize]
{\color{incolor}In[{\color{incolor}34}]:} \PY{o}{\PYZpc{}\PYZpc{}}\PY{k}{tikz} \PYZhy{}p pgfplots \PYZhy{}x \PYZdl{}preamble
         \PYZpc{} had to be this size to have a normal size in latex
             \PYZbs{}begin\PYZob{}axis\PYZcb{}[
                 no markers,
                 domain=0:6,
                 samples=100,
                 ymin=0,
                 axis lines*=left,
                 xlabel=\PYZdl{}x\PYZdl{},
                 ylabel=\PYZdl{}f(x)\PYZdl{},
                 height=5cm,
                 width=12cm,
                 xtick=\PYZbs{}empty,
                 ytick=\PYZbs{}empty,
                 enlargelimits=false,
                 clip=false,
                 axis on top,
                 grid = major,
                 axis lines = middle
               ]
         
             \PYZbs{}def\PYZbs{}mean\PYZob{}3\PYZcb{}
             \PYZbs{}def\PYZbs{}sd\PYZob{}1\PYZcb{}
             \PYZbs{}def\PYZbs{}cilow\PYZob{}\PYZbs{}mean \PYZhy{} 1.96*\PYZbs{}sd\PYZcb{}
             \PYZbs{}def\PYZbs{}cihigh\PYZob{}\PYZbs{}mean + 1.96*\PYZbs{}sd\PYZcb{}
             \PYZbs{}addplot [draw=none, fill=yellow!25, domain=\PYZbs{}cilow:\PYZbs{}cihigh] \PYZob{}gauss(x, \PYZbs{}mean, \PYZbs{}sd)\PYZcb{}
         \PYZbs{}closedcycle;
             \PYZbs{}addplot [very thick,cyan!50!black] \PYZob{}gauss(x, 3, 1)\PYZcb{};
         
             \PYZbs{}pgfmathsetmacro\PYZbs{}valueA\PYZob{}gauss(1,\PYZbs{}mean,\PYZbs{}sd)\PYZcb{}
             \PYZbs{}draw [gray] (axis cs:\PYZbs{}cilow,0) \PYZhy{}\PYZhy{} (axis cs:\PYZbs{}cilow,\PYZbs{}valueA) (axis cs:\PYZbs{}cihigh,0) \PYZhy{}\PYZhy{}
         (axis cs:\PYZbs{}cihigh,\PYZbs{}valueA);
             \PYZbs{}draw [yshift=0.3cm, latex\PYZhy{}latex](axis cs:\PYZbs{}cilow, 0) \PYZhy{}\PYZhy{} node [above] \PYZob{}Area = \PYZdl{}0.95\PYZdl{}\PYZcb{}
         (axis cs:\PYZbs{}cihigh, 0);
         
             \PYZbs{}node[below] at (axis cs:\PYZbs{}cilow, 0)  \PYZob{}\PYZdl{}\PYZbs{}overline\PYZob{}X\PYZcb{} \PYZhy{} 1.96S\PYZdl{}\PYZcb{};
             \PYZbs{}node[below] at (axis cs:\PYZbs{}mean, 0)  \PYZob{}\PYZdl{}\PYZbs{}overline\PYZob{}X\PYZcb{}\PYZdl{}\PYZcb{};
             \PYZbs{}node[below] at (axis cs:\PYZbs{}cihigh, 0)  \PYZob{}\PYZdl{}\PYZbs{}overline\PYZob{}X\PYZcb{} + 1.96S\PYZdl{}\PYZcb{};
         
             \PYZbs{}node[below=0.75cm,text width=4cm] at (axis cs:\PYZbs{}mean, 0)\PYZob{}Sampling Distribution\PYZcb{};
         
         \PYZbs{}end\PYZob{}axis\PYZcb{}
\end{InVerbatim}
    \begin{center}
    \adjustimage{max size={0.9\linewidth}{0.9\paperheight},min size={0.5\linewidth}{!}}{24_confidence_intervals_shallow_examples_files/24_confidence_intervals_shallow_examples_63_0.png}
    \end{center}
    { \hspace*{\fill} \\}
    
    \textbf{\(t\) distribution}:
\begin{InVerbatim}[commandchars=\\\{\},fontsize=\scriptsize]
{\color{incolor}In[{\color{incolor}35}]:} \PY{n}{preamble}\PY{o}{=}\PY{l+s+s1}{\PYZsq{}\PYZsq{}\PYZsq{}}
         \PY{l+s+s1}{    }\PY{l+s+s1}{\PYZbs{}}\PY{l+s+s1}{pgfmathdeclarefunction}\PY{l+s+si}{\PYZob{}gamma\PYZcb{}}\PY{l+s+si}{\PYZob{}1\PYZcb{}}\PY{l+s+s1}{\PYZob{}}\PY{l+s+s1}{\PYZpc{}}
         \PY{l+s+s1}{        }\PY{l+s+s1}{\PYZbs{}}\PY{l+s+s1}{pgfmathparse}\PY{l+s+s1}{\PYZob{}}\PY{l+s+s1}{2.506628274631*sqrt(1/\PYZsh{}1)+ 0.20888568*(1/\PYZsh{}1)\PYZca{}(1.5)+}
         \PY{l+s+s1}{0.00870357*(1/\PYZsh{}1)\PYZca{}(2.5)\PYZhy{} (174.2106599*(1/\PYZsh{}1)\PYZca{}(3.5))/25920\PYZhy{}}
         \PY{l+s+s1}{(715.6423511*(1/\PYZsh{}1)\PYZca{}(4.5))/1244160)*exp((\PYZhy{}ln(1/\PYZsh{}1)\PYZhy{}1)*\PYZsh{}1\PYZcb{}}\PY{l+s+s1}{\PYZpc{}}
         \PY{l+s+s1}{\PYZcb{}}
         
         \PY{l+s+s1}{    }\PY{l+s+s1}{\PYZbs{}}\PY{l+s+s1}{pgfmathdeclarefunction}\PY{l+s+si}{\PYZob{}student\PYZcb{}}\PY{l+s+si}{\PYZob{}2\PYZcb{}}\PY{l+s+s1}{\PYZob{}}\PY{l+s+s1}{\PYZpc{}}
         \PY{l+s+s1}{        }\PY{l+s+s1}{\PYZbs{}}\PY{l+s+s1}{pgfmathparse}\PY{l+s+s1}{\PYZob{}}\PY{l+s+s1}{gamma((\PYZsh{}2+1)/2.)/(sqrt(\PYZsh{}2*pi) *gamma(\PYZsh{}2/2.))}
         \PY{l+s+s1}{*((1+(\PYZsh{}1*\PYZsh{}1)/\PYZsh{}2)\PYZca{}(\PYZhy{}(\PYZsh{}2+1)/2.))\PYZcb{}}\PY{l+s+s1}{\PYZpc{}}
         \PY{l+s+s1}{\PYZcb{}}
         \PY{l+s+s1}{\PYZsq{}\PYZsq{}\PYZsq{}}
\end{InVerbatim}
    ~
\begin{InVerbatim}[commandchars=\\\{\},fontsize=\scriptsize]
{\color{incolor}In[{\color{incolor}36}]:} \PY{o}{\PYZpc{}\PYZpc{}}\PY{k}{tikz} \PYZhy{}p pgfplots \PYZhy{}x \PYZdl{}preamble
         \PYZbs{}begin\PYZob{}axis\PYZcb{}[
                 no markers,
                 domain=\PYZhy{}6:6,
                 samples=100,
                 ymin=0,
                 axis lines*=left,
                 xlabel=\PYZdl{}x\PYZdl{},
                 height=5cm,
                 width=12cm,
                 xtick=\PYZbs{}empty,
                 ytick=\PYZbs{}empty,
                 enlargelimits=false,
                 clip=false,
                 axis on top,
                 grid = major,
                 axis lines = middle,
                 y axis line style=\PYZob{}draw opacity=0.25\PYZcb{}
         ]
             \PYZbs{}def\PYZbs{}mean\PYZob{}0\PYZcb{}
             \PYZbs{}def\PYZbs{}sd\PYZob{}1\PYZcb{}
             \PYZbs{}def\PYZbs{}df\PYZob{}3\PYZcb{}
             \PYZbs{}def\PYZbs{}cilow\PYZob{}\PYZhy{}3.182\PYZcb{}
             \PYZbs{}def\PYZbs{}cihigh\PYZob{}3.182\PYZcb{}
         
             \PYZbs{}addplot [draw=none, fill=yellow!25, domain=\PYZbs{}cilow:\PYZbs{}cihigh] \PYZob{}student(x, \PYZbs{}df)\PYZcb{}
         \PYZbs{}closedcycle;
             \PYZbs{}addplot [very thick,cyan!50!black] \PYZob{}student(x, \PYZbs{}df)\PYZcb{} node [pos=0.6, anchor=mid
         west, xshift=2em, append after command=\PYZob{}(\PYZbs{}tikzlastnode.west) edge [thin, gray]
         +(\PYZhy{}2em,0)\PYZcb{}] \PYZob{}\PYZdl{}df=3\PYZdl{}\PYZcb{};;
         
             \PYZpc{}https://tex.stackexchange.com/questions/453059/pgfmathsetmacro\PYZhy{}creates\PYZhy{}dimensions\PYZhy{}
         too\PYZhy{}large\PYZhy{}for\PYZhy{}t\PYZhy{}distribution/453062
             \PYZbs{}addplot [ycomb, gray, no markers, samples at=\PYZob{}\PYZbs{}cilow, \PYZbs{}cihigh\PYZcb{}] \PYZob{}student(x, \PYZbs{}df)\PYZcb{};
             \PYZbs{}draw [yshift=0.2cm, latex\PYZhy{}latex](axis cs:\PYZbs{}cilow, 0) \PYZhy{}\PYZhy{} node [above] \PYZob{}Area = \PYZdl{}0.95\PYZdl{}\PYZcb{}
         (axis cs:\PYZbs{}cihigh, 0);
         
             \PYZbs{}node[below] at (axis cs:\PYZbs{}cilow, 0)  \PYZob{}\PYZbs{}cilow\PYZcb{};
             \PYZbs{}node[below] at (axis cs:\PYZbs{}mean, 0)  \PYZob{}0\PYZcb{};
             \PYZbs{}node[below] at (axis cs:\PYZbs{}cihigh, 0)  \PYZob{}\PYZbs{}cihigh\PYZcb{};
         
         
             \PYZbs{}node[below=0.75cm,align=center, text width=10cm] at (axis cs:\PYZbs{}mean, 0)\PYZob{}Sampling
         Distribution has \PYZdl{}t\PYZdl{} distribution for low sample sizes\PYZcb{};
         
         \PYZbs{}end\PYZob{}axis\PYZcb{}
\end{InVerbatim}
    \begin{center}
    \adjustimage{max size={0.9\linewidth}{0.9\paperheight},min size={0.5\linewidth}{!}}{24_confidence_intervals_shallow_examples_files/24_confidence_intervals_shallow_examples_67_0.png}
    \end{center}
    { \hspace*{\fill} \\}
    

    % Add a bibliography block to the postdoc
    
    
    
    \end{document}
