
% Default to the notebook output style

    


% Inherit from the specified cell style.




    
\documentclass[float=false,crop=false]{standalone}

    
    
\usepackage{../myipy2tex}  % NOTE WE ARE ASSSUMING THE STYLE FILE TO BE ONE FOLDER ABOVE
\usepackage{../myipy2tex_custom}  % YOUR FURTHER CUSTOM STYLES FOR IPYTHON TO LATEX

% if you need to cross reference to any raw tex file from this resultant tex file you  need to refer them here..
% it is not needed when you compile main.tex but make sure the labels are unique
\ifstandalone
\usepackage{xr-hyper} % Needed for external references
    \externaldocument{24_Hypothesis_Testing_Main} 
\title{Hypothesis Testing}
\fi




    


    


    \begin{document}
    
    
    \maketitle
    
    

    
    As seen earlier in confidence intervals, using Wald's method for the
sample proportions do not yield promising results as widely believed. So
we will only stick to case when conditions are met to make the sampling
distribution normalcy good enough.

    \section{When sample sizes are high}\label{when-sample-sizes-are-high}

Suppose that we have a normal \textbf{sampling distribution} described
by random variable \(\dfrac{Y}{n} = N\Big(p_1, \dfrac{p_1q_1}{n}\Big)\)
created from a population distribution which is a Bernoulli distribution
with mean \(p_1\) and standard deviation \(p_1q_1\). Note that \(Y\)
represents the sum of \emph{successes} in a sample set, and thus
\(\dfrac{Y}{n}\) represents sample proportions. For example, for any
\emph{kth} sample set of \(\dfrac{Y}{n}\), we calculate sample
proportion statistic,
\(\dfrac{Y_{k}}{n} = \dfrac {1}{n} \sum\limits_{i=1}^n Y_{ki}\), where
\(Y_{ki}\) is \(i\)th sample in \(k\)th sample set of sampling
distribution described by \(\dfrac{Y}{n}\). If \(\alpha\) is the
significance level, then we could derive the conditions for hypothesis
testing as follows. Below is our sampling distribution as null
hypothesis, with \(\alpha\) as significance level. This is for alternate
hypothesis being \(H_a: \mu > \mu_{y/n}\) so we consider the right tail
area. One could try similar approach for left or both tails depending on
if \(H_a\) is \(H_a: \mu < \mu_{y/n}\) or \(H_a: \mu \neq \mu_{y/n}\)
respectively.
% remove input part of cells with tag to_remove
    %((- if cell.metadata.hide_input -))% remove input part of cells with tag to_remove
    %((- if cell.metadata.hide_input -))% remove input part of cells with tag to_remove
    %((- if cell.metadata.hide_input -))
    \begin{center}
    \adjustimage{max size={0.9\linewidth}{0.9\paperheight},min size={0.5\linewidth}{!}}{24_HT_1_proportion_files/24_HT_1_proportion_4_0.png}
    \end{center}
    { \hspace*{\fill} \\}
    
    The significance level \(\alpha\), corresponds to the rest of
\(1-\alpha\) area, that is green area as shown above.

    \begin{equation}
    \begin{aligned}
        P\Big(  \frac{Y}{n} \geq \mu_{y/n} + z_{\alpha}\sigma_{y/n} \Big) = \alpha \nonumber \\
        \therefore P\Bigg(  \dfrac{\dfrac{Y}{n} -  \mu_{y/n}}{\sigma_{y/n}} - \geq z_{\alpha} \Bigg) = \alpha \nonumber \\
        P\Bigg(  \dfrac{\dfrac{Y}{n} -  p_1}{ \sqrt{\frac{p_1q_1}{n}} } - \geq z_{\alpha} \Bigg) = \alpha \label{eq:201}
    \end{aligned}
\end{equation}

    Let the z score be,
\(z = \dfrac{\frac{Y}{n} - \mu_{y/n}}{\sigma_{y/n}}\), then
\(P(z \geq z_{\alpha}) = \alpha\)
% remove input part of cells with tag to_remove
    %((- if cell.metadata.hide_input -))
    \begin{center}
    \adjustimage{max size={0.9\linewidth}{0.9\paperheight},min size={0.5\linewidth}{!}}{24_HT_1_proportion_files/24_HT_1_proportion_8_0.png}
    \end{center}
    { \hspace*{\fill} \\}
    
    Our allowed critical region in sampling distribution is
\((\mu_{y/n} + z_{\alpha}\sigma{y/n}, \infty)\), where the probability
of making Type I error is \(\alpha\). Our allowed critical region in
\emph{standardized} sampling distribution would be
\((z_{\alpha}, \infty)\). So if our z score falls within
\((z_{\alpha}, \infty)\), we could reject the null hypothesis. This is
also equivalent to saying, if our sample set proportion \(y/n\) falls
within \((\mu_{y/n} + z_{\alpha}\sigma{y/n}, \infty)\), we could reject
the null hypothesis.
\begin{tcolorbox}[colback=green!5,colframe=green!40!black,title=Conditions]
\begin{itemize}
\item One of the main condition to apply hypothesis testing to sample proportions is to ensure the sampling distribution is normal. This is usually ensured when $(np, nq) > 10$ if not population is already normal. 
\item You see, unlike sample means, there was no $\sigma$ not known case in proportions, because we are testing against hypothesized mean $p_1$, so the associated $\sigma$ would be simply $\sqrt{p_1q_1}$. So $p_1$ is a pre requisite against which we need to test, so that is usually given or implicit in case of one proportion, so no $\sigma$ unknown case arises here. 
\end{itemize}
\end{tcolorbox}
    ~
\subsection*{Example}
    \emph{It was claimed that many commercially manufactured dice are not
fair because the ``spots'' are really indentations, so that, for
example, the 6-side is lighter than the 1-side. To test, in an
experiment, several such dice were rolled, to yield a total of
\(n=8000\) observations, out of which 6 resulted, \(1389\) times. Is
there a significant evidence that dice favor a 6 far more than a fair
die would? Assume \(\alpha = 0.05\)}

    \textbf{Solution:}

Let us assume null hypothesis as a fair die, nothing to doubt about. The
probability of getting a 6 in fair die is \(p=1/6\). So

\(H_0: \mu_{y/n} = p = 1/6\)\\
\(H_a: \mu_{y/n} = p \neq 1/6\)

We have a sample size of \(n=8000\), so
\(np = (8000)(1/6) = 1333 >> 10, nq = (8000)(5/6) = 6666 >> 10\), so our
normal condition is met. If we continue with sample sets of this size,
we would get a good normal sampling distribution \(\dfrac{Y}{n}\)

Our z score is
\(z = \dfrac{\frac{Y}{n} - p_1}{\sqrt{\frac{p_1q_1}{n}}} = \dfrac{(1389/5000) - (1/6)}{\sqrt{\frac{(1/6)(5/6)}{8000}}}\)
\begin{InVerbatim}[commandchars=\\\{\},fontsize=\scriptsize]
{\color{incolor}In[{\color{incolor}11}]:} \PY{n}{Y}\PY{p}{,}\PY{n}{n}\PY{p}{,}\PY{n}{p\PYZus{}1}\PY{p}{,}\PY{n}{q\PYZus{}1} \PY{o}{=} \PY{l+m+mi}{1389}\PY{p}{,} \PY{l+m+mi}{8000}\PY{p}{,} \PY{l+m+mi}{1}\PY{o}{/}\PY{l+m+mi}{6}\PY{p}{,}\PY{l+m+mi}{5}\PY{o}{/}\PY{l+m+mi}{6}
         \PY{n}{num} \PY{o}{=} \PY{p}{(}\PY{n}{Y}\PY{o}{/}\PY{n}{n}\PY{p}{)} \PY{o}{\PYZhy{}} \PY{p}{(}\PY{n}{p\PYZus{}1}\PY{p}{)}
         \PY{k+kn}{from} \PY{n+nn}{math} \PY{k}{import} \PY{n}{sqrt}
         \PY{n}{den} \PY{o}{=} \PY{n}{sqrt}\PY{p}{(}\PY{n}{p\PYZus{}1}\PY{o}{*}\PY{n}{q\PYZus{}1}\PY{o}{/}\PY{n}{n}\PY{p}{)}
         \PY{n}{zs} \PY{o}{=} \PY{n+nb}{round}\PY{p}{(}\PY{n}{num}\PY{o}{/}\PY{n}{den}\PY{p}{,} \PY{l+m+mi}{4}\PY{p}{)}
         \PY{n+nb}{print}\PY{p}{(}\PY{n}{zs}\PY{p}{)}
\end{InVerbatim}
    \begin{Verbatim}[commandchars=\\\{\},fontsize=\footnotesize]
1.67

    \end{Verbatim}

    Our \textbf{allowed} critical region starts from \(z_{0.05} = 1.645\).
The z score \(z = 1.67\) is greater that that, which means, if we select
this sample set as critical region's starting point, our probability of
making Type I error is smaller than allowed \(\alpha = 0.05\). So we
\textbf{reject the null hypothesis} , thus suggesting there is stronger
evidence for alternate \(H_a\).
% remove input part of cells with tag to_remove
    %((- if cell.metadata.hide_input -))
    \begin{center}
    \adjustimage{max size={0.9\linewidth}{0.9\paperheight},min size={0.5\linewidth}{!}}{24_HT_1_proportion_files/24_HT_1_proportion_17_0.png}
    \end{center}
    { \hspace*{\fill} \\}
    
    \section{Conditions Summary}\label{conditions-summary}
				\begin{tikzpicture}[node distance=2cm]
		\node (start) [startstop] {Start};
		\node (dec1) [decision, below of=start, yshift=-1cm] {$(np, nq) > 10?$};
		\node (stop) [startstop, right of=dec1, xshift=3cm] {Stop};
		
		
		\node (pro1) [process, below of=dec1, yshift=-1cm] {Use $z$\\$\newline\displaystyle\sigma_y = \frac{pq}{n}$};

		
        \draw [arrow] (start) -- (dec1);
		\draw [arrow] (dec1) --  node[anchor=east] {yes} node[anchor=south, white, fill=black!30!green,xshift=-1.5cm, yshift=-0.20cm] {PR1} (pro1);
		\draw [arrow] (dec1) --  node[anchor=south, xshift=-0.5cm] {no}  (stop);
		
        \end{tikzpicture}	

    % Add a bibliography block to the postdoc
    
    
    
    \end{document}
