
% Default to the notebook output style

    


% Inherit from the specified cell style.




    
\documentclass[float=false,crop=false]{standalone}

    
    
\usepackage{../myipy2tex}  % NOTE WE ARE ASSSUMING THE STYLE FILE TO BE ONE FOLDER ABOVE
\usepackage{../myipy2tex_custom}  % YOUR FURTHER CUSTOM STYLES FOR IPYTHON TO LATEX

% if you need to cross reference to any raw tex file from this resultant tex file you  need to refer them here..
% it is not needed when you compile main.tex but make sure the labels are unique
\ifstandalone
\usepackage[numbers]{natbib}
\bibliographystyle{abbrvnat}
\usepackage{xr-hyper} % Needed for external references
    \externaldocument{24_Hypothesis_Testing_Main} 
\title{Hypothesis Testing}
\fi




    


    


    \begin{document}
    
    
    \maketitle
    
    

    
    \section{When Successes and Failures are high
enough}\label{when-successes-and-failures-are-high-enough}

\subsection{\texorpdfstring{\((p_1,p_2)\)
known}{(p\_1,p\_2) known}}\label{p_1p_2-known}

Suppose that we are interested in comparing two approximately normal
sampling distributions described by random variables
\(\dfrac{Y_1}{n_1} = N\Big(p_1,\dfrac{p_1q_1}{n_1}\Big)\) and
\(\dfrac{Y_2}{n_2} = N\Big(p_2,\dfrac{p_2q_2}{n_2}\Big)\), created from
population distributions which are Bernoulli distributions. Note that
\(Y_1\) represents the sum of \emph{successes} in a sample set, and thus
\(\dfrac{Y_1}{n_1}\) represents sample proportions. For example, for any
\emph{kth} sample set of \(\dfrac{Y_1}{n_1}\), we calculate sample
proportion statistic,
\(\dfrac{Y_{1k}}{n_1} = \dfrac {1}{n} \sum\limits_{i=1}^n Y_{1ki}\),
where \(Y_{1ki}\) is \(i\)th sample in \(k\)th sample set of sampling
distribution described by \(\dfrac{Y_1}{n_1}\). Similarly for
\(\dfrac{Y_2}{n_2}\). Then, if no of success and failures are high
enough
\footnote{https://www.khanacademy.org/math/ap-statistics/two-sample-inference/two-sample-z-test-proportions/v/hypothesis-test-for-difference-in-proportions},
that is at least \textgreater{} 10, as a general rule, we could assume
that the random variable \(W = \dfrac{Y_1}{n_1} - \dfrac{Y_2}{n_2}\) has
approximately normal distribution \(W = N(p_w, \sigma_w^2)\) where
\(p_w = p_1 - p_2\) and
\(\sigma_w = \sqrt{\dfrac{p_1q_1}{n_1} + \dfrac{p_2q_2}{n_2}}\) and has
shown below, before standardization to Z. We \emph{destandradize} from
Z, because, each \(\alpha\) could be linked to corresponding \(z\)
score, which further could be linked to actual \(w\) or x axis in
question.

    ~
% remove input part of cells with tag to_remove
    %((- if cell.metadata.hide_input -))% remove input part of cells with tag to_remove
    %((- if cell.metadata.hide_input -))% remove input part of cells with tag to_remove
    %((- if cell.metadata.hide_input -))
    \begin{center}
    \adjustimage{max size={0.9\linewidth}{0.9\paperheight},min size={0.5\linewidth}{!}}{24_HT_2_proportions_files/24_HT_2_proportions_4_0.png}
    \end{center}
    { \hspace*{\fill} \\}
    
    The significance level \(\alpha\), corresponds to the rest of
\(1-\alpha\) area, that is green area as shown above.

    \begin{equation}
    \begin{aligned}
        P(W \geq \mu_w + z_{\alpha}\sigma_w) = \alpha \nonumber \\
        \therefore P\Big(\dfrac{W - \mu_w}{\sigma_w} \geq z_{\alpha}\Big) = \alpha \nonumber \\
        P\Bigg(\dfrac{(\frac{Y_1}{n_1} - \frac{Y_2}{n_2}) - (p_1 - p_2)}{\sqrt{ \frac{p_1q_1}{n_1} + \frac{p_2q_2}{n_2}  }} \geq z_{\alpha}\Bigg) = \alpha \label{eq:401}
    \end{aligned}
\end{equation}

    Typically, null hypothesis is \(p_1 = p_2\), so, assigning it to a
common \(p\), i.e \(p1=p2=p\),..

    \begin{equation}
    \begin{aligned}
        P\Bigg(\dfrac{\frac{Y_1}{n_1} - \frac{Y_2}{n_2}}{\sqrt{ \frac{p_1q_1}{n_1} + \frac{p_2q_2}{n_2}  }} \geq z_{\alpha}\Bigg) = \alpha \nonumber \\
        P\Bigg(\dfrac{\frac{Y_1}{n_1} - \frac{Y_2}{n_2}}{\sqrt{ pq (\frac{1}{n_1} + \frac{1}{n_2})  }} \geq z_{\alpha}\Bigg) = \alpha  \label{eq:402}      
    \end{aligned}
\end{equation}

    Thus the z score for given sample data would be
\(z = \dfrac{\frac{Y_1}{n_1} - \frac{Y_2}{n_2}}{\sqrt{ pq (\frac{1}{n_1} + \frac{1}{n_2}) }}\)

    So if our alternate hypothesis is that \(H_a: p_1 > p_2\), then we could
calculate Z score as above and if that is beyond \(z_{\alpha}\) we could
reject null hypothesis.
% remove input part of cells with tag to_remove
    %((- if cell.metadata.hide_input -))
    \begin{center}
    \adjustimage{max size={0.9\linewidth}{0.9\paperheight},min size={0.5\linewidth}{!}}{24_HT_2_proportions_files/24_HT_2_proportions_11_0.png}
    \end{center}
    { \hspace*{\fill} \\}
    
    We could simlarly derive for \(H_a: p_1 < p_2\), and
\(H_a: p_1 \neq p_2\).

    \subsection{\texorpdfstring{\((p_1,p_2)\)
unknown}{(p\_1,p\_2) unknown}}\label{p_1p_2-unknown}

Of course, the above section was for pedagogical purposes, to illustrate
the concept. In reality, the individual \(p_1\) and \(p_2\) are not
hypothesized typically, and usually compared only to see if there is
significant evidence that if one is greater/smaller/different from the
other. In which case we simply could use our best estimator \(\hat{p}\)
for calculating standard deviation in place of \(p\). There are usually
two ways, here.

\paragraph{\texorpdfstring{Way 1: Calculate weighted
\(p\)}{Way 1: Calculate weighted p}}\label{way-1-calculate-weighted-p}

This is usually given as \(\hat{p} = \dfrac{Y_1 + Y_2}{n_1 + n_2}\). And
then, \(\ref{eq:402}\) beocmes

\begin{equation}
    \begin{aligned}
        P\Bigg(\dfrac{\frac{Y_1}{n_1} - \frac{Y_2}{n_2}}{\sqrt{ \hat{p}\hat{q} (\frac{1}{n_1} + \frac{1}{n_2})  }} \geq z_{\alpha}\Bigg) = \alpha 
    \end{aligned}
\end{equation}

At the time of this writing, I could not find a derivation for the same,
so over to next one.

\paragraph{\texorpdfstring{Way 2: Use sample
\(\hat{p_1},\hat{p_2}\)}{Way 2: Use sample \textbackslash{}hat\{p\_1\},\textbackslash{}hat\{p\_2\}}}\label{way-2-use-sample-hatp_1hatp_2}

This is straight forward approach directly from \(\ref{eq:401}\), with
\(p_1 = p_2\)

\begin{equation}
    \begin{aligned}
        P\Bigg(\dfrac{\hat{p_1} - \hat{p_2}}{\sqrt{ \frac{\hat{p_1}\hat{q_1}}{n_1} + \frac{\hat{p_2}\hat{q_2}}{n_2}  }} \geq z_{\alpha}\Bigg) = \alpha \label{eq:404}
    \end{aligned}
\end{equation}
\begin{tcolorbox}[colback=green!5,colframe=green!40!black,title=Tips]
\begin{itemize}
\item Equation $\ref{eq:404}$ would be the one mostly used for almost any of difference of proportions problems (of course adapted to right or left or both tails as needed)
\end{itemize}
\end{tcolorbox}
    \paragraph{Example}\label{example}

A machine shop that manufactures toggle levers has both a day and a
night shift. A toggle lever is defective if a standard nut cannot be
screwed onto the threads. Let p1 and p2 be the proportion of defective
levers among those manufactured by the day and night shifts,
respectively. We shall test the null hypothesis, \(H_0: p_1 = p_2\),
against a two-sided alternative hypothesis based on two random samples,
each of 1000 levers taken from the production of the respective shifts.

\textbf{(a)} Define the test statistic and a critical region that has an
\(\alpha = 0.05\) significance level. Sketch a standard normal pdf
illustrating this critical region.

\textbf{(b)} If \(y_1 = 37\) and \(y_2 = 53\) defectives were observed
for the day and night shifts, respectively, calculate the value of the
test statistic. Locate the calculated test statistic on your figure in
part (a) and state your conclusion.

This example was taken from exercise \emph{8.3-11} in \citet{robert2015}

    \textbf{Solution:}

Day:
\(y_1 = 37, n_1 = 1000, \hat{p_1} = \dfrac{Y_1}{n_1} = \dfrac{37}{1000} = 0.037\)\\
Night:
\(y_2 = 53, n_2 = 1000, \hat{p_2} = \dfrac{Y_2}{n_2} = \dfrac{53}{1000} = 0.053\)

\textbf{(a)}

It is said, "two sided alternative hypothesis", so below is our required
test statistic. note we have used our best estimators
\((\hat{p_1},\hat{p_2})\) so result is only approximate.
% remove input part of cells with tag to_remove
    %((- if cell.metadata.hide_input -))
    \begin{center}
    \adjustimage{max size={0.9\linewidth}{0.9\paperheight},min size={0.5\linewidth}{!}}{24_HT_2_proportions_files/24_HT_2_proportions_17_0.png}
    \end{center}
    { \hspace*{\fill} \\}
    
    Calculating the values \(\sigma_w\), we could arrive at
\(w = \mu_w \pm z_{\alpha/2}\sigma_w = \pm z_{\alpha/2}\sigma_w\) value
beyond which we could define critical region \(\alpha\). Since it is
double tailed, we already know \(z_{0.025} = 1.96\).

\(\sigma_w = \sqrt{\frac{\hat{p_1}\hat{q_1}}{n_1} + \frac{\hat{p_2}\hat{q_2}}{n_2}} = \sqrt{\frac{(0.037)(1-0.037)}{1000} + \frac{(0.053)(1-0.053)}{1000}}\)
\begin{InVerbatim}[commandchars=\\\{\},fontsize=\scriptsize]
{\color{incolor}In[{\color{incolor}14}]:} \PY{n}{p\PYZus{}1\PYZus{}hat}\PY{p}{,} \PY{n}{q\PYZus{}1\PYZus{}hat}\PY{p}{,} \PY{n}{p\PYZus{}2\PYZus{}hat}\PY{p}{,} \PY{n}{q\PYZus{}2\PYZus{}hat}\PY{p}{,} \PY{n}{n\PYZus{}1}\PY{p}{,} \PY{n}{n\PYZus{}2} \PY{o}{=} \PY{l+m+mf}{0.037}\PY{p}{,} \PY{l+m+mi}{1}\PY{o}{\PYZhy{}}\PY{l+m+mf}{0.037}\PY{p}{,} \PY{l+m+mf}{0.053}\PY{p}{,} \PY{l+m+mi}{1}\PY{o}{\PYZhy{}}\PY{l+m+mf}{0.053}\PY{p}{,} \PY{l+m+mi}{1000}\PY{p}{,}
         \PY{l+m+mi}{1000}
         \PY{n}{z\PYZus{}0025} \PY{o}{=} \PY{l+m+mf}{1.96}
         
         \PY{k+kn}{from} \PY{n+nn}{math} \PY{k}{import} \PY{n}{sqrt}
         \PY{n}{s\PYZus{}w} \PY{o}{=} \PY{n}{sqrt}\PY{p}{(} \PY{p}{(}\PY{n}{p\PYZus{}1\PYZus{}hat}\PY{o}{*}\PY{n}{q\PYZus{}1\PYZus{}hat}\PY{o}{/}\PY{n}{n\PYZus{}1}\PY{p}{)} \PY{o}{+} \PY{p}{(}\PY{n}{p\PYZus{}2\PYZus{}hat}\PY{o}{*}\PY{n}{q\PYZus{}2\PYZus{}hat}\PY{o}{/}\PY{n}{n\PYZus{}2}\PY{p}{)} \PY{p}{)}
         \PY{n+nb}{print}\PY{p}{(}\PY{n}{s\PYZus{}w}\PY{o}{*}\PY{n}{z\PYZus{}0025}\PY{p}{)}
\end{InVerbatim}
    \begin{Verbatim}[commandchars=\\\{\},fontsize=\footnotesize]
0.018157472158866164

    \end{Verbatim}
% remove input part of cells with tag to_remove
    %((- if cell.metadata.hide_input -))
    \begin{center}
    \adjustimage{max size={0.9\linewidth}{0.9\paperheight},min size={0.5\linewidth}{!}}{24_HT_2_proportions_files/24_HT_2_proportions_20_0.png}
    \end{center}
    { \hspace*{\fill} \\}
    
    We could alreay take a call on our null hypothesis, Our
\(\hat{p_1} - \hat{p_2} = 0.037 - 0.053 = 0.016 < 0.0018\), so we cannot
reject \(H_0\). Our standardized test statistic would be simply z
distribution as below.
% remove input part of cells with tag to_remove
    %((- if cell.metadata.hide_input -))
    \begin{center}
    \adjustimage{max size={0.9\linewidth}{0.9\paperheight},min size={0.5\linewidth}{!}}{24_HT_2_proportions_files/24_HT_2_proportions_22_0.png}
    \end{center}
    { \hspace*{\fill} \\}
    
    \textbf{(b)}

We have already kinda finished the solution, but for question's sake we
could complete it fully by calculating the Z score.

Using \(\ref{eq:404}\),
\(z = \dfrac{\hat{p_1} - \hat{p_2}}{\sqrt{ \frac{\hat{p_1}\hat{q_1}}{n_1} + \frac{\hat{p_2}\hat{q_2}}{n_2} }}\)
\begin{InVerbatim}[commandchars=\\\{\},fontsize=\scriptsize]
{\color{incolor}In[{\color{incolor}19}]:} \PY{n}{num} \PY{o}{=} \PY{n}{p\PYZus{}1\PYZus{}hat} \PY{o}{\PYZhy{}} \PY{n}{p\PYZus{}2\PYZus{}hat}
         \PY{n}{den} \PY{o}{=} \PY{n}{s\PYZus{}w}
         \PY{n}{num}\PY{o}{/}\PY{n}{den}
\end{InVerbatim}
\begin{Verbatim}[commandchars=\\\{\}]
{\color{outcolor}Out[{\color{outcolor}19}]:} -1.7271126578424703
\end{Verbatim}
            
    Being double tailed operation, our z score is thus \(\pm 1.727\). And
since \(\pm 1.727 < \pm 1.96\), we again \textbf{cannot reject null
hypothesis} because then our probability of making Type I error would be
more than allowed limit of \(\alpha = 0.05\). Our standardized test
statistic, with \(\pm z_{\alpha/2} = \pm 1.727\) is shown below.
% remove input part of cells with tag to_remove
    %((- if cell.metadata.hide_input -))
    \begin{center}
    \adjustimage{max size={0.9\linewidth}{0.9\paperheight},min size={0.5\linewidth}{!}}{24_HT_2_proportions_files/24_HT_2_proportions_26_0.png}
    \end{center}
    { \hspace*{\fill} \\}
    
    Though visibly not clear, one could use z table to find that
\(z_{1.727}\) takes more area than 0.05 which corresponds to
\(z_{1.96}\). Thus we conclude our answer.

    \section{Conditions Summary}\label{conditions-summary}
				\begin{tikzpicture}[node distance=2cm]
		\node (start) [startstop] {Start};
		\node (dec1) [decision, below of=start, yshift=-1cm] {${\scriptstyle (Y_1, Y_2) > 10? }$\\$\scriptstyle{(1-Y_1,1-Y_2) > 10?}$};
		\node (stop) [startstop, right of=dec1, xshift=3cm] {Stop};
		
		
		\node (pro1) [process, below of=dec1, yshift=-2cm] {Use $z$\\$\newline\displaystyle\sigma_w = \sqrt{ \frac{\hat{p_1}\hat{q_1}}{n_1} + \frac{\hat{p_2}\hat{q_2}}{n_2}  }$};

		
        \draw [arrow] (start) -- (dec1);
		\draw [arrow] (dec1) --  node[anchor=east] {yes} node[anchor=south, white, fill=black!30!green,xshift=-1.5cm, yshift=-0.20cm] {PR1} (pro1);
		\draw [arrow] (dec1) --  node[anchor=south, xshift=-0.5cm] {no}  (stop);
		
        \end{tikzpicture}	% if references to be checked during stand alone..
    \ifstandalone
	\clearpage	
	\bibliography{../references} 
	\fi

    % Add a bibliography block to the postdoc
    
    
    
    \end{document}
