
% Default to the notebook output style

    


% Inherit from the specified cell style.




    
\documentclass[float=false,crop=false]{standalone}

    
    
\usepackage{../myipy2tex}  % NOTE WE ARE ASSSUMING THE STYLE FILE TO BE ONE FOLDER ABOVE
\usepackage{../myipy2tex_custom}  % YOUR FURTHER CUSTOM STYLES FOR IPYTHON TO LATEX

% if you need to cross reference to any raw tex file from this resultant tex file you  need to refer them here..
% it is not needed when you compile main.tex but make sure the labels are unique
\ifstandalone
\usepackage[numbers]{natbib}
\bibliographystyle{abbrvnat}
\usepackage{xr-hyper} % Needed for external references
    \externaldocument{29_appendix}
    \externaldocument{29_MLE_Regression_Main} 
\title{Hypothesis Testing}
\fi




    


    


    \begin{document}
    
    
    \maketitle
    
    

    
    \section{\texorpdfstring{\(e\) and natural
logarithms}{e and natural logarithms}}\label{e-and-natural-logarithms}

\subsection{\texorpdfstring{The basics of
\(e\)}{The basics of e}}\label{the-basics-of-e}

\paragraph{Case 1:}\label{case-1}

Suppose we have a function \(M(t) = 2^t\) and we are interested in its
rate of change \(\dfrac{dM(t)}{dt}\).
% remove input part of cells with tag to_remove
    %((- if cell.metadata.hide_input -))% remove input part of cells with tag to_remove
    %((- if cell.metadata.hide_input -))% remove input part of cells with tag to_remove
    %((- if cell.metadata.hide_input -))
    \begin{center}
    \adjustimage{max size={0.9\linewidth}{0.9\paperheight},min size={0.5\linewidth}{!}}{29_appendix_1_files/29_appendix_1_3_0.png}
    \end{center}
    { \hspace*{\fill} \\}
    
    \[
\dfrac{dM(t)}{dt} =  \lim_{dt \to 0} \dfrac{2^{(t+dt)}-2^t}{dt} =  \lim_{dt \to 0} \dfrac{2^t2^{dt} - 2^t}{dt} 
=  \lim_{dt \to 0} \dfrac{2^t(2^{dt} - 1)}{dt} \\
\therefore \dfrac{d(2^t)}{dt} = \lim_{dt \to 0} 2^t\Big(\dfrac{2^{dt}-1}{dt}\Big) 
\]

    One could note that, as \(dt \to 0\), the component
\(\Big( \dfrac{2^{dt}-1}{dt} \Big) \to 0.6931\) as shown below.
\begin{InVerbatim}[commandchars=\\\{\},fontsize=\scriptsize]
{\color{incolor}In[{\color{incolor}36}]:} \PY{n}{dt} \PY{o}{=} \PY{p}{[}\PY{l+m+mf}{0.1}\PY{p}{,} \PY{l+m+mf}{0.01}\PY{p}{,} \PY{l+m+mf}{0.005}\PY{p}{,} \PY{l+m+mf}{0.001}\PY{p}{,}\PY{l+m+mf}{0.0005}\PY{p}{,} \PY{l+m+mf}{0.0001}\PY{p}{]}
         \PY{n}{cl} \PY{o}{=} \PY{p}{[}\PY{n+nb}{print}\PY{p}{(}\PY{n}{i}\PY{p}{,} \PY{n+nb}{round}\PY{p}{(}\PY{p}{(}\PY{l+m+mi}{2}\PY{o}{*}\PY{o}{*}\PY{n}{i}\PY{o}{\PYZhy{}}\PY{l+m+mi}{1}\PY{p}{)}\PY{o}{/}\PY{n}{i}\PY{p}{,}\PY{l+m+mi}{5}\PY{p}{)}\PY{p}{)} \PY{k}{for} \PY{n}{i} \PY{o+ow}{in} \PY{n}{dt}\PY{p}{]}
\end{InVerbatim}
    \begin{Verbatim}[commandchars=\\\{\},fontsize=\footnotesize]
0.1 0.71773
0.01 0.69556
0.005 0.69435
0.001 0.69339
0.0005 0.69327
0.0001 0.69317

    \end{Verbatim}

    So,

\begin{equation}
    \dfrac{d(2^t)}{dt} = \lim_{dt \to 0} 2^t\Big(\dfrac{2^{dt}-1}{dt}\Big) = 2^t(0.6931)
\end{equation}

    \paragraph{Case 2:}\label{case-2}

Suppose we have a function \(M(t) = 3^t\) and we are interested in its
rate of change \(\dfrac{dM(t)}{dt}\).

    \[
\dfrac{dM(t)}{dt} =  \lim_{dt \to 0} \dfrac{3^{(t+dt)}-3^t}{dt} =  \lim_{dt \to 0} \dfrac{3^t2^{dt} - 3^t}{dt} 
=  \lim_{dt \to 0} \dfrac{3^t(2^{dt} - 1)}{dt} \\
\therefore \dfrac{d(3^t)}{dt} = \lim_{dt \to 0} 3^t\Big(\dfrac{3^{dt}-1}{dt}\Big)
\]

    One could note that, as \(dt \to 0\), the component
\(\Big( \dfrac{3^{dt}-1}{dt} \Big) \to 1.09867\) as shown below.
\begin{InVerbatim}[commandchars=\\\{\},fontsize=\scriptsize]
{\color{incolor}In[{\color{incolor}38}]:} \PY{n}{dt} \PY{o}{=} \PY{p}{[}\PY{l+m+mf}{0.1}\PY{p}{,} \PY{l+m+mf}{0.01}\PY{p}{,} \PY{l+m+mf}{0.005}\PY{p}{,} \PY{l+m+mf}{0.001}\PY{p}{,}\PY{l+m+mf}{0.0005}\PY{p}{,} \PY{l+m+mf}{0.0001}\PY{p}{]}
         \PY{n}{cl} \PY{o}{=} \PY{p}{[}\PY{n+nb}{print}\PY{p}{(}\PY{n}{i}\PY{p}{,} \PY{n+nb}{round}\PY{p}{(}\PY{p}{(}\PY{l+m+mi}{3}\PY{o}{*}\PY{o}{*}\PY{n}{i}\PY{o}{\PYZhy{}}\PY{l+m+mi}{1}\PY{p}{)}\PY{o}{/}\PY{n}{i}\PY{p}{,}\PY{l+m+mi}{5}\PY{p}{)}\PY{p}{)} \PY{k}{for} \PY{n}{i} \PY{o+ow}{in} \PY{n}{dt}\PY{p}{]}
\end{InVerbatim}
    \begin{Verbatim}[commandchars=\\\{\},fontsize=\footnotesize]
0.1 1.16123
0.01 1.10467
0.005 1.10164
0.001 1.09922
0.0005 1.09891
0.0001 1.09867

    \end{Verbatim}

    So,

\begin{equation}
    \dfrac{d(3^t)}{dt} = \lim_{dt \to 0} 3^t\Big(\dfrac{3^{dt}-1}{dt}\Big) = 3^t(1.09867)
\end{equation}

    \paragraph{Generalization}\label{generalization}

Similarly for any \(M(t) = a^t\), we could prove,

\begin{equation}
    \dfrac{d(a^t)}{dt} = \lim_{dt \to 0} a^t\Big(\dfrac{a^{dt}-1}{dt}\Big) = a^tC \ \ \text{where C is some constant}
\end{equation}

    \paragraph{Wonder}\label{wonder}

Naturally if we wonder, is there any similar \(M(t)\) for which the
derivative is itself? (In other words, that some constant becomes 1?!).
We could solve this as below.

We want to find \(a\) such that,\\
\[
\lim_{dt \to 0} \Big(\dfrac{a^{dt}-1}{dt}\Big) = 1
\]

Rewriting, \[
\lim_{dt \to 0} a^{dt} = 1 + dt \\
\therefore a = \lim_{dt \to 0} (1 + dt)^{1/dt} \\
\]

    We can mathematically prove that, \((1+n)^{1/n}\) approaches a constant,
but for here, we could simply compute like earlier, what is the value it
is approaching..
\begin{InVerbatim}[commandchars=\\\{\},fontsize=\scriptsize]
{\color{incolor}In[{\color{incolor}45}]:} \PY{n}{dt} \PY{o}{=} \PY{p}{[}\PY{l+m+mf}{0.1}\PY{p}{,} \PY{l+m+mf}{0.01}\PY{p}{,} \PY{l+m+mf}{0.005}\PY{p}{,} \PY{l+m+mf}{0.001}\PY{p}{,}\PY{l+m+mf}{0.0005}\PY{p}{,} \PY{l+m+mf}{0.0001}\PY{p}{]}
         \PY{n}{cl} \PY{o}{=} \PY{p}{[}\PY{n+nb}{print}\PY{p}{(}\PY{n}{i}\PY{p}{,} \PY{n+nb}{round}\PY{p}{(}\PY{p}{(}\PY{l+m+mi}{1} \PY{o}{+} \PY{n}{i}\PY{p}{)}\PY{o}{*}\PY{o}{*}\PY{p}{(}\PY{l+m+mi}{1}\PY{o}{/}\PY{n}{i}\PY{p}{)}\PY{p}{,}\PY{l+m+mi}{5}\PY{p}{)}\PY{p}{)} \PY{k}{for} \PY{n}{i} \PY{o+ow}{in} \PY{n}{dt}\PY{p}{]}
\end{InVerbatim}
    \begin{Verbatim}[commandchars=\\\{\},fontsize=\footnotesize]
0.1 2.59374
0.01 2.70481
0.005 2.71152
0.001 2.71692
0.0005 2.7176
0.0001 2.71815

    \end{Verbatim}

    \(\therefore\) we do have one constant \(2.718\) for which, the
derivative of it is itself. That is,
\begin{tcolorbox}[colback=green!5,colframe=green!40!black,title=The value of $e$]
Let $e$ = 2.71815, then
\begin{equation}
    \dfrac{d(e^t)}{dt} = e^t  \label{eq:MA01}
\end{equation}
\end{tcolorbox}
    \subsection{\texorpdfstring{Derivative of
\(e^{ct}\)}{Derivative of e\^{}\{ct\}}}\label{derivative-of-ect}

What is \(\dfrac{d(e^{ct})}{dt}\)? This can be solved by chain rule in
differential calculus.

Let \(u = ct\), then by chain rule,

\[
\dfrac{d(e^{ct})}{dt} = \dfrac{d(e^u)}{dt} = \dfrac{d(e^u)}{du}\dfrac{du}{dt} = e^u\dfrac{du}{dt}
\]

Substituting \(u=ct\) back, \[
\dfrac{d(e^{ct})}{dt} =  e^{ct}\dfrac{d(ct)}{dt} = ce^{ct}
\]
\begin{tcolorbox}[colback=green!5,colframe=green!40!black,title=The derivative of $e^{ct}$]
Let $e$ = 2.71815, then
\begin{equation}
    \dfrac{d(e^{ct})}{dt} = ce^{ct}  \label{eq:MA02}
\end{equation}
\end{tcolorbox}
    \subsection{\texorpdfstring{Using \(e\) for any exponent
form}{Using e for any exponent form}}\label{using-e-for-any-exponent-form}

A short summary of what we saw earlier.

\[
\begin{aligned}
& \dfrac{d(2^t)}{dt} = (0.6931)2^t \\
& \dfrac{d(3^t)}{dt} = (1.0986)3^t \\
& \dfrac{d(a^t)}{dt} = (C)a^t , \ \ \ \ \text{where $C$ is some constant depending on $a$} \\
\end{aligned}
\]

Let \(2 = e^C\). Then

\[
2^t = e^{Ct}
\]

Taking derivatives on both sides, \[
\dfrac{d(2^t)}{dt} = \dfrac{ d(e^{Ct}) }{dt} \\
\implies (0.6931)2^t = Ce^{Ct} \\
\implies C = 0.6931
\]

That is, the constant we earlier got, is nothing but the power to which
we need to raise \(e\) to get the base value \(2\). That is,
\(2 = e^{0.6931}\). We could call this constant as \textbf{natural
logarithm of 2}, denoted by \(ln(2)\) or \(log_e(2)\)

Similarly, let \(3 = e^C\). Then

\[
3^t = e^{Ct}
\]

Taking derivatives on both sides, \[
\dfrac{d(3^t)}{dt} = \dfrac{ d(e^{Ct}) }{dt} \\
\implies (1.0986)2^t = Ce^{Ct} \\
\implies C = 1.0986
\]

Thus, \(3 = e^{1.0986}\). We could call this constant as \textbf{natural
logarithm of 3}, denoted by \(ln(3)\) or \(log_e(3)\)

Summarizing,

\[
\begin{aligned}
& 2 = e^{ln(2)}, \ \ ln(2) = log_e(2) = 0.6931 \\
& 3 = e^{ln(3)}, \ \ ln(3) = log_e(3) = 1.0986 \\
\end{aligned}
\]
\begin{tcolorbox}[colback=green!5,colframe=green!40!black,title=Any number in terms of $e$]
Any number could be equated by $e$ to the power of its natural logarithmic value, which is a unique constant that could be derived. 
\begin{equation}
    a = e^{ln(a)}, \ \ ln(a) = log_e(a)  \label{eq:MA03}
\end{equation}
\end{tcolorbox}
    \subsection{Multiplication and Division
simplified}\label{multiplication-and-division-simplified}

Suppose we have a function \(L(p,q) = p^yq^z\)

We could make the multiplication of such exponents in to simpler form of
addition of their natural logarithms as below.

Let \(p = e^{C_1}\) and \(q = e^{C_2}\), then we already have seen,
\(C_1 = ln(p), C_2 = ln(q)\).

\[
\therefore p^yq^z = e^{C_1y}e^{C_2}z = e^{C_1y + C_2z} = e^{ln(p)y + ln(q)z}
\]

If \(L = e^{ln(L)}\) similarly, then we could write,

\[
\begin{aligned}
L = p^yq^z \\
e^{ln(L)} = e^{ln(p)y + ln(q)z} \\
\implies ln(L) = yln(p) + zln(q) \ \ \ \ \text{or} \\
log_e(L) = ylog_e(p) + zlog_e(q)
\end{aligned}
\]

If \(L(p,q) = \dfrac{p^y}{q^z}\)

\[
\dfrac{p^y}{q^z} = \dfrac{e^{C_1y}}{e^{C_2z}} = e^{C_1y - C_2z} = e^{ln(p)y - ln(q)z}
\]

If \(L = e^{ln(L)}\) similarly, then we could write,

\[
\begin{aligned}
L = \dfrac{p^y}{q^z} \\
e^{ln(L)} = e^{ln(p)y - ln(q)z} \\
\implies ln(L) = yln(p) - zln(q) \ \ \ \ \text{or} \\
log_e(L) = ylog_e(p) - zlog_e(q)
\end{aligned}
\]

Thus we have simplified multiplication and division to addition and
subtraction provided we know the equivalent natural logarithms of the
values involved.
\begin{tcolorbox}[colback=green!5,colframe=green!40!black,title=Multiplication and Divison Simplification]
\begin{itemize}
\item If $L(p,q) = p^yq^z$, then 
\begin{equation}
    log_e(L) = ylog_e(p) + zlog_e(q)  \label{eq:MA04}
\end{equation}
\item If $L(p,q) = \dfrac{p^y}{q^z}$, then 
\begin{equation}
    log_e(L) = ylog_e(p) - zlog_e(q)  \label{eq:MA05}
\end{equation}
\end{itemize}
\end{tcolorbox}
    \subsection{\texorpdfstring{Derivatives of
\(ln\)}{Derivatives of ln}}\label{derivatives-of-ln}

We only see few derivatives that could be useful in MLE.

Q1: What is the derivative of \(\dfrac{d(ln(x))}{dx}\)?

Let \(y = ln(x) = log_ex\). This means, \(e^y = x\)

Differentiating that,

\[
e^y = x \\
\dfrac{d(e^y)}{dx} = \dfrac{dx}{dx} \\
e^y\dfrac{dy}{dx} = 1 \\
\dfrac{dy}{dx} = \dfrac{1}{e^y} \\
\therefore \dfrac{d(ln(x))}{dx} = \dfrac{1}{x}
\]

    Q2: What is the derivative of \(\dfrac{d(ln(1 - x))}{dx}\)?

Let \(y = ln(1 - x) = log_e(1-x)\). This means, \(e^y = 1 - x\)

Differentiating that,

\[
e^y = 1 - x \\
\dfrac{d(e^y)}{dx} = \dfrac{d(1-x)}{dx} \\
e^y\dfrac{dy}{dx} = -1 \\
\dfrac{dy}{dx} = \dfrac{-1}{e^y} = \dfrac{-1}{1 - x} \\
\therefore \dfrac{d(ln(1-x))}{dx} = \dfrac{-1}{1 - x}
\]

    Q3: What is the derivative of \(\dfrac{d(ln(cx))}{dx}\)?

Let \(u=cx\), \(y = ln(u)\). This means, \(e^y = u\)

Differentiating that,

\[
\begin{aligned}
e^y = u \\
\dfrac{d(e^y)}{dx} = \dfrac{du}{dx} = \dfrac{d(2\pi x)}{dx} = 2\pi \\
e^y\dfrac{dy}{dx} = 2\pi \\
\dfrac{dy}{dx} = \dfrac{2\pi}{e^y} = \dfrac{2\pi}{u} = \dfrac{2\pi}{2\pi x} = \dfrac{1}{x}
\end{aligned}
\]
\begin{tcolorbox}[colback=green!5,colframe=green!40!black,title=Derivatives of $ln$]
\begin{itemize}
\item 
\begin{equation}
    \dfrac{d(ln(x))}{dx} = \dfrac{1}{x} \label{eq:MA06}
\end{equation}
\item 
\begin{equation}
    \dfrac{d(ln(1-x))}{dx} = \dfrac{-1}{1 - x} \label{eq:MA07}
\end{equation}
\item 
\begin{equation}
    \dfrac{d(ln(cx))}{dx} = \dfrac{1}{x} \label{eq:MA08}
\end{equation}
\end{itemize}
\end{tcolorbox}

    % Add a bibliography block to the postdoc
    
    
    
    \end{document}
