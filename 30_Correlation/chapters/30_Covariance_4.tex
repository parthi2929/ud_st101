
% Default to the notebook output style

    


% Inherit from the specified cell style.




    
\documentclass[float=false,crop=false]{standalone}

    
    


% if you need to cross reference to any raw tex file from this resultant tex file you  need to refer them here..
% it is not needed when you compile main.tex but make sure the labels are unique
\ifstandalone
\usepackage{../myipy2tex}  % NOTE WE ARE ASSSUMING THE STYLE FILE TO BE ONE FOLDER ABOVE
\usepackage{../myipy2tex_custom}  % YOUR FURTHER CUSTOM STYLES FOR IPYTHON TO LATEX
\usepackage{../mytikz_custom}
\usepackage{xr-hyper} % Needed for external references   
    \externaldocument{30_Covariance_1} 
    \externaldocument{30_Covariance_2} 
    \externaldocument{30_Covariance_3} 
    \externaldocument{30_Correlation_Main} 
\title{Covariance and Correlation}
\fi




    


    


    \begin{document}
    
    
    \maketitle
    
    

    % remove input part of cells with tag to_remove
    %((- if cell.metadata.hide_input -))% remove input part of cells with tag to_remove
    %((- if cell.metadata.hide_input -))
    \section{Generalization}\label{generalization}

So far we have seen Covariance for discrete X, Y random variables. This
could easily be transferred to continuous variables as well. However
before generalization of the formula, we need to generalize the way the
sample set is provided as well.

Suppose the sample set is given as
\((X,Y) = (x_1,y_1), (x_2, y_2), (x_3, y_3) \cdots (x_N, y_N)\) then, if
we say equi probable, then \(p(X,Y)\) could be simply tabulated in
different ways depending on the function \(h(X,Y)\) that is, if we take
the deformed or standard formula. This is illustrated in figure
\(\ref{fig:C4_001}\).
\begin{figure}
\centering
\begin{tikzpicture}
\tikzset{%
square matrix/.style={
    matrix of nodes,
    column sep=-\pgflinewidth, 
    row sep=-\pgflinewidth,
    nodes in empty cells,
    nodes={draw,
      minimum size=#1,
      anchor=center,
      align=center,
      inner sep=0pt
    },
    column 1/.style={nodes={fill=green!20}},
    row 7/.style={nodes={fill=green!20}},
  },
  square matrix/.default=0.9cm
}
\newcommand{\tx}{|[fill=cyan!20]|}
\newcommand{\tb}{|[fill=yellow!20]|}
\newcommand{\tg}{|[fill=gray!20]|}

\matrix[square matrix] (A)
{
$y_6$ & \tx  & \tx  & \tx  & \tx  & \tx  & \tg $\dfrac{1}{N}$ \\ 
$y_5$ & \tx  & \tx  & \tx  & \tx  & \tg $\dfrac{1}{N}$ & \tb  \\
$y_4$ & \tx  & \tx  & \tx  & \tg $\dfrac{1}{N}$ & \tb  & \tb  \\
$y_3$ & \tx  & \tx  & \tg $\dfrac{1}{N}$ & \tb  &  \tb & \tb    \\ 
$y_2$ & \tx  & \tg $\dfrac{1}{N}$ & \tb  & \tb  & \tb  & \tb  \\ 
$y_1$ & \tg $\dfrac{1}{N}$ & \tb & \tb  & \tb  & \tb  & \tb  \\ 
  & $x_1$ & $x_2$ & $x_3$ & $x_4$ & $x_5$ & $x_6$\\ 
};

\draw (A-7-1.north east)--(A-7-1.south west);
\node[below left=4mm and 1mm of A-7-1.north east] {$x$};
\node[above right=4mm and 1mm of A-7-1.south west] {$y$};

        \node[below=0.50cm,align=center,text width=5cm] at (0,-3)
        {
            Plot A \\$p(X,Y)$ for standard formula \\ $h(X,Y) = (X - \overline{X})(Y - \overline{Y})$ \\  \phantom{ }
        };

\end{tikzpicture}
\qquad
\begin{tikzpicture}
\tikzset{%
square matrix/.style={
    matrix of nodes,
    column sep=-\pgflinewidth, 
    row sep=-\pgflinewidth,
    nodes in empty cells,
    nodes={draw,
      minimum size=#1,
      anchor=center,
      align=center,
      inner sep=0pt
    },
    column 1/.style={nodes={fill=green!20}},
    row 7/.style={nodes={fill=green!20}},
  },
  square matrix/.default=0.9cm
}
\newcommand{\tx}{|[fill=cyan!20]|}
\newcommand{\tb}{|[fill=yellow!20]|}
\newcommand{\tg}{|[fill=gray!20]|}

\matrix[square matrix] (A)
{
$y_6$ & \tx $\dfrac{1}{N^2}$ & \tx$\dfrac{1}{N^2}$ & \tx $\dfrac{1}{N^2}$ & \tx $\dfrac{1}{N^2}$ & \tx $\dfrac{1}{N^2}$ & \tg $\dfrac{1}{N^2}$ \\ 
$y_5$ & \tx $\dfrac{1}{N^2}$ & \tx $\dfrac{1}{N^2}$ & \tx $\dfrac{1}{N^2}$ & \tx $\dfrac{1}{N^2}$ & \tg $\dfrac{1}{N^2}$ & \tb $\dfrac{1}{N^2}$  \\
$y_4$ & \tx $\dfrac{1}{N^2}$ & \tx $\dfrac{1}{N^2}$ & \tx $\dfrac{1}{N^2}$ & \tg $\dfrac{1}{N^2}$ & \tb $\dfrac{1}{N^2}$ & \tb $\dfrac{1}{N^2}$ \\
$y_3$ & \tx $\dfrac{1}{N^2}$ & \tx $\dfrac{1}{N^2}$ & \tg $\dfrac{1}{N^2}$ & \tb $\dfrac{1}{N^2}$ & \tb $\dfrac{1}{N^2}$ & \tb 
$\dfrac{1}{N^2}$ \\ 
$y_2$ & \tx $\dfrac{1}{N^2}$ & \tg $\dfrac{1}{N^2}$ & \tb $\dfrac{1}{N^2}$ & \tb $\dfrac{1}{N^2}$ & \tb $\dfrac{1}{N^2}$ & \tb $\dfrac{1}{N^2}$ \\ 
$y_1$ & \tg $\dfrac{1}{N^2}$ & \tb $\dfrac{1}{N^2}$ & \tb $\dfrac{1}{N^2}$ & \tb $\dfrac{1}{N^2}$ & \tb $\dfrac{1}{N^2}$ & \tb $\dfrac{1}{N^2}$ \\ 
  & $x_1$ & $x_2$ & $x_3$ & $x_4$ & $x_5$ & $x_6$\\ 
};

\draw (A-7-1.north east)--(A-7-1.south west);
\node[below left=4mm and 1mm of A-7-1.north east] {$x$};
\node[above right=4mm and 1mm of A-7-1.south west] {$y$};

        \node[below=0.50cm,align=center,text width=5cm] at (0,-3)
        {
            Plot B \\$p(X,Y)$ for deformed formula \\ $h(X_i,Y_i,X_j,Y_j) = (X_i - X_j)(Y_i - Y_j)$
        };

\end{tikzpicture}
\caption{$p(X,Y)$ depending on $h(X,Y)$} \label{fig:C4_001}
\end{figure}
    This was simply because, of the way we indexed the sample points. In
Plot A, we do not have a \((x_2, y_1)\), because we just numbered as
\((x_1,y_1), (x_2, y_2), (x_3, y_3) \cdots (x_N, y_N)\), and it worked
because standard formula needed only one time indexing via \(i\). But in
Plot B, we had double indexing via \(i,j\), this is why the probability
at each \emph{cell} also became \(1/N^2\). Most often we do not use the
deformed formula and stick to standard formula. Further, often the given
probability density function (if given), would be something like this.
% remove input part of cells with tag to_remove
    %((- if cell.metadata.hide_input -))
    \begin{center}
    \adjustimage{max size={0.9\linewidth}{0.9\paperheight},min size={0.5\linewidth}{!}}{30_Covariance_4_files/30_Covariance_4_5_0.pdf}
    \end{center}
    { \hspace*{\fill} \\}
    
    Here our indexing style has to differ. Now we have
\((x_1,x_2) = (100,250)\) and \((y_1,y_2,y_3) = (0,100,200)\). If we
line up these sample pairs, we get

\[
(x_1, y_1), (x_1, y_2), (x_1, y_3), (x_2, y_1), (x_2, y_2), (x_2, y_3)
\]

Thus even with standard formula due to data being in a different format,
we would need to use double summation in order to vary i and j to
different limits separately. Thus naturally our standard formula would
become

    \[
\mathrm{Cov}(X,Y) = \sum\limits_{i=1}^2\sum\limits_{j=1}^3(x_i - \overline{x})(y_j - \overline{y})p(x_i , y_i)
\]

Generalizing the standard formula, and also extending to continuous X
and Y, we could say,

\begin{tcolorbox}[colback=green!5,colframe=green!40!black,title=Generalized Standard Covariance Formula]
The \textbf{covariance} between two rv's X and Y is
\begin{equation}
\begin{aligned}
    \mathrm{Cov}(X,Y) = E[(X - \mu_x)(Y - \mu_Y)] \\
    = \begin{cases}
    \sum\limits_{x}\sum\limits_{y}(x - \mu_x)(y - \mu_y)p(x,y) & \text{X,Y discrete} \\ \\
    \int_{-\infty}^{\infty}\int_{-\infty}^{\infty}(x - \mu_x)(y - \mu_y)f(x,y)dxdy & \text{X,Y continuous}
    \end{cases}
    \label{eq:C4_001}
\end{aligned}
\end{equation}
\end{tcolorbox}
    Depending on samples are from population or we deal with entire
population, either \(\overline{x}\) or \(\mu_X\) could be used
respectively.

    \section{Example}\label{example}

We have already explained the concept with an example, so here will see
a different approach.

Suppose joint and marginal pmf's for X = automobile policy deductible
amount and Y = homeowner policy deductible amount are as below. Find the
covariance.
% remove input part of cells with tag to_remove
    %((- if cell.metadata.hide_input -))
    \begin{center}
    \adjustimage{max size={0.9\linewidth}{0.9\paperheight},min size={0.5\linewidth}{!}}{30_Covariance_4_files/30_Covariance_4_11_0.pdf}
    \end{center}
    { \hspace*{\fill} \\}
    This example was taken from \citet{devore2011}
    Since we need the means in the equation, let us calculate them first.

\[
\mu_x = \sum\limits_{i=1}^2x_ip_X(x_i) = 100(0.5) + 250(0.5) = 175 \\
\mu_y = \sum\limits_{i=1}^2y_ip_Y(y_i) = 0(0.25) + 100(0.25) + 200(0.5) = 125
\]

    Coming to Covariance,

\[\begin{aligned}
\mathrm{Cov}(X,Y) = \sum\limits_{i=1}^{2}\sum\limits_{j=1}^{3}(x_i - \mu_x)(y_j - \mu_y)p(x_i,y_j) \\
= (x_1 - 175)(y_1 - 125)p(x_1,y_1) + (x_1 - 175)(y_2 - 125)p(x_1,y_2) + (x_1 - 175)(y_3 - 125)p(x_1,y_3) \\
+ (x_2 - 175)(y_1 - 125)p(x_2,y_1) + (x_2 - 175)(y_2 - 125)p(x_2,y_2) + (x_2 - 175)(y_3 - 125)p(x_2,y_3) \\ \\
= (100 - 175)(0 - 125)p(100,0) + (100 - 175)(100 - 125)p(100,100) + (100 - 175)(200 - 125)p(100,200) \\
+ (250 - 175)(0 - 125)p(250,0) + (250 - 175)(100 - 125)p(250,100) + (250 - 175)(200 - 125)p(250,200) \\ \\
= (100 - 175)(0 - 125)0.20 + (100 - 175)(100 - 125)0.10 + (100 - 175)(200 - 125)0.20 \\
+ (250 - 175)(0 - 125)0.05 + (250 - 175)(100 - 125)0.15 + (250 - 175)(200 - 125)0.30 \\ \\
= 1875
\end{aligned}\]

    What just happpened? How come we took all possible pairs of \((x,y)\)
given in joing pmf as \emph{samples}? Earlier, when we visualized TIA
for random samples, we assumed that \(h(X,Y)\) had equal probability for
all of its values, thus resulting in a constant probability for entire
summation. So it was enough if we look at it from the sky or top or
whatever. If the probability density in the summation is a variable,
then just by looking at 2D, we are missing the \emph{contribution} of
pmf to the summation. Now that we have varying pmf for different pairs
of \(x,y\), we need to account for that, because pairs having higher
probability will attract more samples than those that would not, thus
potentially forming a relationship between X and Y. This is evident the
moment we visualize in 3D as shown in figure \(\ref{fig:C4_002}\). In
3D, it is evident now, the green has more volume, than red, so we could
expect higher samples in these region than the red, thus suggesting in
fact a \emph{positive} correlation. Thus, yeah it is no more just a TIA
,but \textbf{total interested volume, TIV}. Also, a pmf resembles all
possible values of \((x,y)\), so could imagine, sample set of all
possible values in any multiples (1 occurance per pair, or 10 occurance
per pair, etc).

\begin{figure}
\centering% remove input part of cells with tag to_remove
    %((- if cell.metadata.hide_input -))
    \begin{center}
    \adjustimage{max size={0.9\linewidth}{0.9\paperheight},min size={0.5\linewidth}{!}}{30_Covariance_4_files/30_Covariance_4_17_0.pdf}
    \end{center}
    { \hspace*{\fill} \\}
    \caption{Figure 6: The Visualization of standard formula in 2D and 3D} \label{fig:C4_002}
\end{figure}
\begin{tcolorbox}[colback=green!5,colframe=green!40!black,title=Generalized Standard Covariance Visualization]
The better generalized visualization of standard covariance formula is in volume, if underlying joint probability density function is not a constant.
\begin{equation}
\begin{aligned}
    \mathrm{Cov}(X,Y) = \sum\limits_{x}\sum\limits_{y}(x_i - \overline{x})(y_i - \overline{y})p(x_i,y_i) \\
    = (x_1 - \overline{x})(y_1 - \overline{y})p(x_1,y_1) + \cdots + 
    (x_i - \overline{x})(y_i - \overline{y})p(x_i,y_i) + \cdots \\
    = V_{11} + \cdots + V_{ij} + \cdots 
    \label{eq:C4_002}
\end{aligned}
\end{equation}
\end{tcolorbox}

    % Add a bibliography block to the postdoc
    
    
    
    \end{document}
