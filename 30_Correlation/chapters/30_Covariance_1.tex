
% Default to the notebook output style

    


% Inherit from the specified cell style.




    
\documentclass[float=false,crop=false]{standalone}

    
    


% if you need to cross reference to any raw tex file from this resultant tex file you  need to refer them here..
% it is not needed when you compile main.tex but make sure the labels are unique
\ifstandalone
\usepackage{../myipy2tex}  % NOTE WE ARE ASSSUMING THE STYLE FILE TO BE ONE FOLDER ABOVE
\usepackage{../myipy2tex_custom}  % YOUR FURTHER CUSTOM STYLES FOR IPYTHON TO LATEX
\usepackage{../mytikz_custom}
\usepackage{xr-hyper} % Needed for external references       
    \externaldocument{30_Correlation_Main} 
\title{Covariance and Correlation}
\fi




    


    


    \begin{document}
    
    
    \maketitle
    
    

    % remove input part of cells with tag to_remove
    %((- if cell.metadata.hide_input -))% remove input part of cells with tag to_remove
    %((- if cell.metadata.hide_input -))
    \section{Why}\label{why}

Earlier in regression, we said, by eyeballing, one could roughly
conclude if a viable regression line possible that could be useful. But
that of course, is not a rigorous approach to decide upon the
\textbf{goodness} of relation between two variables. Note that for all
below variation in X and Y, we could still draw a regression line, but
it is obvious, for those \textbf{closer} to linear relationship between
them positively or negatively will benefit from regression line than
those who do not.
% remove input part of cells with tag to_remove
    %((- if cell.metadata.hide_input -))
    \begin{center}
    \adjustimage{max size={0.9\linewidth}{0.9\paperheight},min size={0.5\linewidth}{!}}{30_Covariance_1_files/30_Covariance_1_3_0.pdf}
    \end{center}
    { \hspace*{\fill} \\}
    
    \section{What}\label{what}

\paragraph{Relationship Definition}\label{relationship-definition}

Let X and Y be the random variables involved, and each point
representing a \((x,y)\) pair value. What we want to see is, how is each
point located with respect to every other point in the given sample set.
Also we want to know if that is in a positive or negative way. Imagine a
pair of points \((x_1,y_1)\) and \((x_2,y_2)\). Let \(x_1\) and \(x_2\)
be in increasing order, then if \((y_2 > y_1)\) we could say, the pair
is in a positive relationship. We could also sort \(y_1,y_2,\cdots\) in
increasing order, and then say if \(x_2 > x_1\), then the pair is in a
positive relationship. By positive we just mean, with increasing \(x\)
the \(y\) increases. The negative relationship is defined simply the
opposite of it, that is, with increasing \(x\), the \(y\) decreases. Or
with increasing \(y\), the \(x\) decreases. Consequently, in terms of
points we could say, given \(y_1<y_2\) , if \(x_1 > x_2\), then its a
negative relationship. Summarizing we could stick to below convention,
but one could try the alternate also.

Given \((x_1, y_1),(x_2, y_2)\) and \(y\) is in increasing order, i.e.,
\((y_1 < y_2)\),\\
if \((x_1 < x_2)\) or \((x_2 - x_1>0)\), this implies \(x\) has
increased with \(y\), a positive relationship\\
if \((x_1 > x_2)\) or \((x_1 - x_2>0)\), this implies \(x\) has
decreased with \(y\), a negative relationship

    \paragraph{Visual Quantification via Colored
Rectangles}\label{visual-quantification-via-colored-rectangles}

Now that we have defined the relationship, next should think about
quantification. After all, what we seek is a \emph{measure}, a
quantification of the relationship. How could we quantitatively
differentiate the defined relation between pairs say,
\([(x_1,y_1),(x_2,y_2)]\) and \([(x_3,y_3),(x_4,y_4)]\)? This could be
approached with geometry. Imagine drawing a rectangle based on
\([(x_1,y_1),(x_2,y_2)]\), say \(R_{12}\) and \([(x_3,y_3),(x_4,y_4)]\),
say \(R_{34}\) separately. Then one rectangle's area would be smaller or
larger than the other, indicating a quantified measure of how farther
apart the points are comparitively. Also, we could color the area to
indicate if the involved pair that is used to construct the rectangle is
in a positive or negative relationship. To construct a rectangle out of
two points \([(x_1,y_1),(x_2,y_2)]\), we could just consider them as a
two oppositing corners of the rectangle, and simply draw one whose sides
are parallel to the axes. Let us color green for a positive relationship
and red for a negative relationship. Such a visual quantification is
illustrated below. Note that, a certain transparency is maintained for
each rectangle, so the overlapping does not hide any information, but
simply transparent to us.
% remove input part of cells with tag to_remove
    %((- if cell.metadata.hide_input -))% remove input part of cells with tag to_remove
    %((- if cell.metadata.hide_input -))
    \begin{center}
    \adjustimage{max size={0.9\linewidth}{0.9\paperheight},min size={0.5\linewidth}{!}}{30_Covariance_1_files/30_Covariance_1_7_0.pdf}
    \end{center}
    { \hspace*{\fill} \\}
    
    \begin{equation}
\text{Area of a rectangle,    } R_{ij} = (x_i - x_j)(y_i - y_j)  \label{eq:C001}
\end{equation}

    \section{Area Distribution}\label{area-distribution}

Of course we have not drawn all possible combinations above for given
set of points \((x_1,y_1), (x_2, y_2), (x_3, y_3), (x_4,y_4)\) to
establish first the basic idea, but that is what we would do for any
given set of points:Plot all such relationship rectangles for every
point with every other point in the given sample. We want to know for
each and every point in given set, its relationship with every other
point, \emph{quantitatively}. However there is a problem.

    \begin{quote}
\emph{If we try to plot for every possible pair of given set of data,
there will be symmetrically distributed duplicity which not only
introduces redundancy in the measure, but also neutrlizes our
visualization}
\end{quote}

    That is, if \((x_i,y_i)\) is a positive relationship with \((x_j,y_j)\),
it also means \((x_j,y_j)\) is in negative relationship with
\(x_i, y_i\) in other direction. Trying to take all possible rectangle
will have this duality for all rectangles. For example, by iterated dual
looping if at one iteration if \((x_i,y_i) = (1,2), (x_j,y_j) = (3,4)\),
then down the line, when j takes i value, we have,
\((x_i,y_i) = (3,4), (x_j,y_j = (1,2))\). In terms of rectangle
notation, for every \(R_{ij}\), there is \(R_{ji}\) which is of equal
value.
% remove input part of cells with tag to_remove
    %((- if cell.metadata.hide_input -))
    \begin{center}
    \adjustimage{max size={0.9\linewidth}{0.9\paperheight},min size={0.5\linewidth}{!}}{30_Covariance_1_files/30_Covariance_1_12_0.pdf}
    \end{center}
    { \hspace*{\fill} \\}
    
    Thus the flaw in the visualization already strongly suggests not to take
all rectangles for the measure but may be, just half of it as
representative of entire sample set. Below are the total number of
rectangles for \(N=6\) pairs of sample sets. The blue shaded is
symmetrical to yellow shaded. This is why the measure would be
inherently doubled if all rectangles are taken into account. By nature,
it is not needed. Think about it. Taking all possible rectangles, simply
means, looking for a linear relationship in one direction and then
again, in reverse, and deciding that the relationship is null. We should
instead decide to take in to account only one direction,which means,
only half of below rectangles would sufficely give a measure of
relationship in one direction. Also note the diagonal rectangles have
zero area, thus can be neglected too.
% remove input part of cells with tag to_remove
    %((- if cell.metadata.hide_input -))% remove input part of cells with tag to_remove
    %((- if cell.metadata.hide_input -))
    \begin{center}
    \adjustimage{max size={0.9\linewidth}{0.9\paperheight},min size={0.5\linewidth}{!}}{30_Covariance_1_files/30_Covariance_1_15_0.pdf}
    \end{center}
    { \hspace*{\fill} \\}
    
    Thus, we would just go with only either blue or yellow rectangles as
illustrated above. Let us look closer at the product
\((x_i-x_j)(y_i-y_j)\) for all rectangles. The no of rectangles in the
half we are interested in is given by \(\dfrac{N(N-1)}{2}\). If \(N=6\),
you could observe we have \(\dfrac{(6)(5)}{2} = 15\) rectangles as our
interest out of \(N^2 = 6^2 = 36\) rectangles.

If we untangle the rectangle information systematically,we could come up
with a summation to calculate the total value as below. Let us consider
the \emph{yellow} rectangles (you could try the blue ones)

\begin{itemize}
\tightlist
\item
  Let \(i=1\), then
  \(R_{12} + R_{13} + R_{14} + R_{15} + R_{16} = \sum\limits_{j=i+1}^6R_{1j}\)\\
\item
  Let \(i=2\), then
  \(R_{23} + R_{24} + R_{25} + R_{26} = \sum\limits_{j=i+1}^6R_{2j}\)\\
\item
  Let \(i=3\), then
  \(R_{34} + R_{35} + R_{36} = \sum\limits_{j=i+1}^6R_{3j}\)\\
\item
  Let \(i=4\), then \(R_{45} + R_{46} = \sum\limits_{j=i+1}^6R_{4j}\)\\
\item
  Let \(i=5\), then \(R_{56} = \sum\limits_{j=i+1}^6R_{5j}\)
\end{itemize}

We could thus consilidate the total area of our interest as,

\[
\text{Total Interested Area, TIA} = \sum\limits_{i=1}^5\sum\limits_{j=i+1}^{6}R_{ij}
\]

When \(i=6\), \(j=i+1=7\), and there is no \(R_{67}\), or
\(R_{67} = 0\), so we could rewrite slightly as,

\[
\text{TIA } = \sum\limits_{i=1}^6\sum\limits_{j=i+1}^{6}R_{ij}
\]

    Using \(\ref{eq:C001}\), and generalizing to \(N\),

\begin{equation}
    \text{TIA} = \sum\limits_{i=1}^N\sum\limits_{j=i+1}^{N}(x_i - x_j)(y_i - y_j) \label{eq:C002}
\end{equation}

    \emph{Alternate approach}: We instead could have taken all area, and
then simply divided by 2. Here, the derivation is straight forward. For
\(N=6\), there are \(N^2=36\) rectangles possible. And as indexed in
last diagram, the total area would be,

\[
\text{Total Area} = \sum\limits_{i=1}^N\sum\limits_{j=1}^NR_{ij}
\]

Using \(\ref{eq:C001}\) and taking the half as that is our interested
area, we get,

    \begin{equation}
\text{TIA} = \dfrac{1}{2}\sum\limits_{i=1}^N\sum\limits_{j=1}^{N}(x_i - x_j)(y_i - y_j) \label{eq:C003}
\end{equation}

    Both \(\ref{eq:C002}\) and \(\ref{eq:C003}\) are equivalent, but
\(\ref{eq:C002}\) gives a better intuition, what we are after. Let us
take a closer look next at the rectangular area distribution.


    % Add a bibliography block to the postdoc
    
    
    
    \end{document}
