
% Default to the notebook output style

    


% Inherit from the specified cell style.




    
\documentclass[float=false,crop=false]{standalone}

    
    


% if you need to cross reference to any raw tex file from this resultant tex file you  need to refer them here..
% it is not needed when you compile main.tex but make sure the labels are unique
\ifstandalone
\usepackage{../myipy2tex}  % NOTE WE ARE ASSSUMING THE STYLE FILE TO BE ONE FOLDER ABOVE
\usepackage{../myipy2tex_custom}  % YOUR FURTHER CUSTOM STYLES FOR IPYTHON TO LATEX
\usepackage{../mytikz_custom}
\usepackage{xr-hyper} % Needed for external references   
    \externaldocument{30_Covariance_1} 
    \externaldocument{30_Correlation_Main} 
\title{Covariance and Correlation}
\fi




    


    


    \begin{document}
    
    
    \maketitle
    
    

    % remove input part of cells with tag to_remove
    %((- if cell.metadata.hide_input -))% remove input part of cells with tag to_remove
    %((- if cell.metadata.hide_input -))
    \paragraph{Case 1 : Perfectly positively linearly related dataset
(whew!)}\label{case-1-perfectly-positively-linearly-related-dataset-whew}

Suppose we have such a case as below. you could note, this is of line
\(y=2x\)

\begin{itemize}
\tightlist
\item
  X = 1,2,3,4,5,6\\
\item
  Y = 2,4,6,8,10,12
\end{itemize}

Then, every possible rectangle for each pair of
\([ (x_i, y_i) , (x_j,y_j) ]\) is tabulated below. This illustrates the
redundancy better. Note the repetitive values symmetrically spread from
the diagonal lines. The color gradient gives a better perspective of the
spread. The actual plot of the sample set is given on the right side.
% remove input part of cells with tag to_remove
    %((- if cell.metadata.hide_input -))
    \begin{center}
    \adjustimage{max size={0.9\linewidth}{0.9\paperheight},min size={0.5\linewidth}{!}}{30_Covariance_2_files/30_Covariance_2_3_0.pdf}
    \end{center}
    { \hspace*{\fill} \\}
    
    Using \(\ref{eq:C002}\) or \(\ref{eq:C003}\), TIA for given sample set,
turns out to be \(210\)
\begin{InVerbatim}[commandchars=\\\{\},fontsize=\scriptsize]
{\color{incolor}In[{\color{incolor}33}]:} \PY{n}{X} \PY{p}{,} \PY{n}{Y}\PY{o}{=} \PY{p}{[}\PY{l+m+mi}{1}\PY{p}{,}\PY{l+m+mi}{2}\PY{p}{,}\PY{l+m+mi}{3}\PY{p}{,}\PY{l+m+mi}{4}\PY{p}{,}\PY{l+m+mi}{5}\PY{p}{,}\PY{l+m+mi}{6}\PY{p}{]}\PY{p}{,}\PY{p}{[}\PY{l+m+mi}{2}\PY{p}{,}\PY{l+m+mi}{4}\PY{p}{,}\PY{l+m+mi}{6}\PY{p}{,}\PY{l+m+mi}{8}\PY{p}{,}\PY{l+m+mi}{10}\PY{p}{,}\PY{l+m+mi}{12}\PY{p}{]}
         \PY{n}{N} \PY{o}{=} \PY{n+nb}{len}\PY{p}{(}\PY{n}{X}\PY{p}{)}
         
         \PY{k}{def} \PY{n+nf}{get\PYZus{}TIA}\PY{p}{(}\PY{n}{X}\PY{p}{,}\PY{n}{Y}\PY{p}{)}\PY{p}{:}
             \PY{n}{N} \PY{o}{=} \PY{n+nb}{len}\PY{p}{(}\PY{n}{X}\PY{p}{)}
             \PY{n}{comb\PYZus{}l}\PY{p}{,} \PY{n}{area} \PY{o}{=} \PY{n+nb}{sorted}\PY{p}{(}\PY{n+nb}{zip}\PY{p}{(}\PY{n}{X}\PY{p}{,}\PY{n}{Y}\PY{p}{)}\PY{p}{,} \PY{n}{key}\PY{o}{=}\PY{k}{lambda} \PY{n}{x}\PY{p}{:} \PY{n}{x}\PY{p}{[}\PY{l+m+mi}{1}\PY{p}{]}\PY{p}{)}\PY{p}{,} \PY{l+m+mi}{0}  \PY{c+c1}{\PYZsh{}sorting w.r.t Y}
             \PY{k}{for} \PY{n}{i} \PY{o+ow}{in} \PY{n+nb}{range}\PY{p}{(}\PY{l+m+mi}{0}\PY{p}{,}\PY{n}{N}\PY{p}{)}\PY{p}{:}  \PY{c+c1}{\PYZsh{} equivalent for i = 1 to N because, range is 0 to N\PYZhy{}1}
                 \PY{k}{for} \PY{n}{j} \PY{o+ow}{in} \PY{n+nb}{range}\PY{p}{(}\PY{n}{i}\PY{o}{+}\PY{l+m+mi}{1}\PY{p}{,}\PY{n}{N}\PY{p}{)}\PY{p}{:}
                     \PY{n}{X1}\PY{p}{,} \PY{n}{Y1}\PY{p}{,} \PY{n}{X2}\PY{p}{,} \PY{n}{Y2} \PY{o}{=} \PY{n}{comb\PYZus{}l}\PY{p}{[}\PY{n}{i}\PY{p}{]}\PY{p}{[}\PY{l+m+mi}{0}\PY{p}{]}\PY{p}{,} \PY{n}{comb\PYZus{}l}\PY{p}{[}\PY{n}{i}\PY{p}{]}\PY{p}{[}\PY{l+m+mi}{1}\PY{p}{]}\PY{p}{,} \PY{n}{comb\PYZus{}l}\PY{p}{[}\PY{n}{j}\PY{p}{]}\PY{p}{[}\PY{l+m+mi}{0}\PY{p}{]}\PY{p}{,} \PY{n}{comb\PYZus{}l}\PY{p}{[}\PY{n}{j}\PY{p}{]}\PY{p}{[}\PY{l+m+mi}{1}\PY{p}{]}
                     \PY{n}{d1}\PY{p}{,} \PY{n}{d2} \PY{o}{=} \PY{n}{X2} \PY{o}{\PYZhy{}} \PY{n}{X1}\PY{p}{,} \PY{n}{Y2} \PY{o}{\PYZhy{}} \PY{n}{Y1}
                     \PY{n}{area} \PY{o}{+}\PY{o}{=} \PY{n}{d1}\PY{o}{*}\PY{n}{d2}
             \PY{k}{return} \PY{n}{area}
         
         \PY{n+nb}{print}\PY{p}{(}\PY{n}{get\PYZus{}TIA}\PY{p}{(}\PY{n}{X}\PY{p}{,}\PY{n}{Y}\PY{p}{)}\PY{p}{)}
\end{InVerbatim}
    \begin{Verbatim}[commandchars=\\\{\},fontsize=\footnotesize]
210

    \end{Verbatim}

    \paragraph{Case 2 : Perfectly negatively linearly related
dataset}\label{case-2-perfectly-negatively-linearly-related-dataset}

Suppose we have such a case as below. you could note, this is of line
\(y=14 - 2x\)

\begin{itemize}
\tightlist
\item
  X = 1,2,3,4,5,6\\
\item
  Y = 12,10,8,6,4,2
\end{itemize}

For this, let us check the rectangles' area.
% remove input part of cells with tag to_remove
    %((- if cell.metadata.hide_input -))
    \begin{center}
    \adjustimage{max size={0.9\linewidth}{0.9\paperheight},min size={0.5\linewidth}{!}}{30_Covariance_2_files/30_Covariance_2_7_0.pdf}
    \end{center}
    { \hspace*{\fill} \\}
    
    We see something interesting here. If you might have thought, some way
the rectangular area also represented actual plot could notice it here
that the area plot on left hand side is still similarly symmetrical as
before, even though the plot is perfectly negatively related as shown on
RHS. This is because, that was its definition in first place. The
rectangular area plot on LHW just gives a measure of the spread of
relationship, while the plot on RHS represents the actual location. Also
note, that again, due to symmetry, we have duplicate values, thus
suggesting to halve the measure. And the values are negative. This is
good, now that could help to indicate our sample sets are negatively
linearity related. Let us check out the TIA.
\begin{InVerbatim}[commandchars=\\\{\},fontsize=\scriptsize]
{\color{incolor}In[{\color{incolor}35}]:} \PY{n}{X} \PY{p}{,} \PY{n}{Y}\PY{o}{=} \PY{p}{[}\PY{l+m+mi}{1}\PY{p}{,}\PY{l+m+mi}{2}\PY{p}{,}\PY{l+m+mi}{3}\PY{p}{,}\PY{l+m+mi}{4}\PY{p}{,}\PY{l+m+mi}{5}\PY{p}{,}\PY{l+m+mi}{6}\PY{p}{]}\PY{p}{,}\PY{p}{[}\PY{l+m+mi}{12}\PY{p}{,}\PY{l+m+mi}{10}\PY{p}{,}\PY{l+m+mi}{8}\PY{p}{,}\PY{l+m+mi}{6}\PY{p}{,}\PY{l+m+mi}{4}\PY{p}{,}\PY{l+m+mi}{2}\PY{p}{]}
         \PY{n+nb}{print}\PY{p}{(}\PY{n}{get\PYZus{}TIA}\PY{p}{(}\PY{n}{X}\PY{p}{,}\PY{n}{Y}\PY{p}{)}\PY{p}{)}
\end{InVerbatim}
    \begin{Verbatim}[commandchars=\\\{\},fontsize=\footnotesize]
-210

    \end{Verbatim}

    Its negative. We are already getting somewhere! Let us consider another
case, where there is no linear relationship.

\paragraph{Case 3 : Dataset with no linear
relationship}\label{case-3-dataset-with-no-linear-relationship}

Suppose we have such a case as below.

\begin{itemize}
\tightlist
\item
  X = 1,2,3,4,5,6\\
\item
  Y = 12,10,8,8,10,12
\end{itemize}

The respective rectangle area and plots are as below.
% remove input part of cells with tag to_remove
    %((- if cell.metadata.hide_input -))
    \begin{center}
    \adjustimage{max size={0.9\linewidth}{0.9\paperheight},min size={0.5\linewidth}{!}}{30_Covariance_2_files/30_Covariance_2_11_0.pdf}
    \end{center}
    { \hspace*{\fill} \\}
    
    Again, irrespective of actual plot, the area graph on LHS, is still
symmetrical if you look carefully, assuring, no matter what, the measure
is available in doubled quantity across all possible rectangles, so good
golly gosh, we chose half of the rectangles. Note the RHS plot, there is
clearly not a possibility of a best fit linearity between X and Y, and
this should reflect in our measure. Let us calculate the TIA.
\begin{InVerbatim}[commandchars=\\\{\},fontsize=\scriptsize]
{\color{incolor}In[{\color{incolor}37}]:} \PY{n}{X} \PY{p}{,} \PY{n}{Y} \PY{o}{=} \PY{p}{[}\PY{l+m+mi}{1}\PY{p}{,}\PY{l+m+mi}{2}\PY{p}{,}\PY{l+m+mi}{3}\PY{p}{,}\PY{l+m+mi}{4}\PY{p}{,}\PY{l+m+mi}{5}\PY{p}{,}\PY{l+m+mi}{6}\PY{p}{]}\PY{p}{,}\PY{p}{[}\PY{l+m+mi}{12}\PY{p}{,}\PY{l+m+mi}{10}\PY{p}{,}\PY{l+m+mi}{8}\PY{p}{,}\PY{l+m+mi}{8}\PY{p}{,}\PY{l+m+mi}{10}\PY{p}{,}\PY{l+m+mi}{12}\PY{p}{]}
         \PY{n+nb}{print}\PY{p}{(}\PY{n}{get\PYZus{}TIA}\PY{p}{(}\PY{n}{X}\PY{p}{,}\PY{n}{Y}\PY{p}{)}\PY{p}{)}
\end{InVerbatim}
    \begin{Verbatim}[commandchars=\\\{\},fontsize=\footnotesize]
0

    \end{Verbatim}

    Understandly it is 0. The no linear relationship in a literal sense has
been transformed to a number via our TIA. Recall,

\begin{itemize}
\tightlist
\item
  for a perfectly positively linearly related dataset, we got +210
\item
  for a perfectly negatively linearly related dataset, we got -210
\item
  for a perfectly not linearly related dataset, we got 0
\end{itemize}

Thus our TIA is already proving to be a good measure. Note that, if we
had taken all rectangles and getting 420,-420,0 instead, it would be an
unnecessarily doubled stretch, giving a doubled sense of actual linearly
underneath. By halving the area, that is via TIA, we have taken the
linearity sense in a kind of \emph{same} scale of what it is.

    \paragraph{Case 4: A practical realistic
dataset}\label{case-4-a-practical-realistic-dataset}

Suppose we have such a case as below.

\begin{itemize}
\tightlist
\item
  X = 2.2, 2.7, 3, 3.55, 4, 4.5, 4.75, 5.5
\item
  Y = 14, 23, 13, 22, 15, 20, 28 , 23
\end{itemize}

The respective rectangle area and plots are as below.
% remove input part of cells with tag to_remove
    %((- if cell.metadata.hide_input -))
    \begin{center}
    \adjustimage{max size={0.9\linewidth}{0.9\paperheight},min size={0.5\linewidth}{!}}{30_Covariance_2_files/30_Covariance_2_16_0.pdf}
    \end{center}
    { \hspace*{\fill} \\}
    
    You see, even for a realistic sample set which has some linearity
associated in either direction (positive or negative), the LHS area
diagram has a symmetry as usual. This would always be the case, thus we
are right in taking the half no of rectangles, no matter what the
linearity is. Proceeding to TIA, we get it as
\begin{InVerbatim}[commandchars=\\\{\},fontsize=\scriptsize]
{\color{incolor}In[{\color{incolor}39}]:} \PY{n}{X} \PY{p}{,} \PY{n}{Y} \PY{o}{=} \PY{p}{[}\PY{l+m+mf}{2.2}\PY{p}{,} \PY{l+m+mf}{2.7}\PY{p}{,} \PY{l+m+mi}{3}\PY{p}{,} \PY{l+m+mf}{3.55}\PY{p}{,} \PY{l+m+mi}{4}\PY{p}{,} \PY{l+m+mf}{4.5}\PY{p}{,} \PY{l+m+mf}{4.75}\PY{p}{,} \PY{l+m+mf}{5.5}\PY{p}{]}\PY{p}{,}\PY{p}{[} \PY{l+m+mi}{14}\PY{p}{,} \PY{l+m+mi}{23}\PY{p}{,} \PY{l+m+mi}{13}\PY{p}{,} \PY{l+m+mi}{22}\PY{p}{,} \PY{l+m+mi}{15}\PY{p}{,} \PY{l+m+mi}{20}\PY{p}{,} \PY{l+m+mi}{28} \PY{p}{,} \PY{l+m+mi}{23}\PY{p}{]}
         \PY{n+nb}{print}\PY{p}{(}\PY{n}{get\PYZus{}TIA}\PY{p}{(}\PY{n}{X}\PY{p}{,}\PY{n}{Y}\PY{p}{)}\PY{p}{)}
\end{InVerbatim}
    \begin{Verbatim}[commandchars=\\\{\},fontsize=\footnotesize]
184.39999999999995

    \end{Verbatim}


    % Add a bibliography block to the postdoc
    
    
    
    \end{document}
