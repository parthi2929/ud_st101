
% Default to the notebook output style

    


% Inherit from the specified cell style.




    
\documentclass[float=false,crop=false]{standalone}

    
    


% if you need to cross reference to any raw tex file from this resultant tex file you  need to refer them here..
% it is not needed when you compile main.tex but make sure the labels are unique
\ifstandalone
\usepackage{../myipy2tex}  % NOTE WE ARE ASSSUMING THE STYLE FILE TO BE ONE FOLDER ABOVE
\usepackage{../myipy2tex_custom}  % YOUR FURTHER CUSTOM STYLES FOR IPYTHON TO LATEX
\usepackage{../mytikz_custom}
\usepackage{xr-hyper} % Needed for external references   
    \externaldocument{30_Covariance_1} 
    \externaldocument{30_Correlation_Main} 
\title{Covariance and Correlation}
\fi




    


    


    \begin{document}
    
    
    \maketitle
    
    

    % remove input part of cells with tag to_remove
    %((- if cell.metadata.hide_input -))% remove input part of cells with tag to_remove
    %((- if cell.metadata.hide_input -))
    \begin{Verbatim}[commandchars=\\\{\},fontsize=\footnotesize]
The tikzmagic extension is already loaded. To reload it, use:
  \%reload\_ext tikzmagic

    \end{Verbatim}

    \section{Visualization}\label{visualization}

Now that we have seen TIA is already doing a good job on giving us a
measure of the linearity, we shall come to the core of this section. We
have not yet visualized the totality of the rectangles. We could have
done this earlier, but I wanted to instill a strong sense of what
rectangles are we dealing with and why they are whole representative of
the dataset though we have taken only half of all possible rectangles.
We initially decided how do we color the rectangles, based on positive
or negative relationship as a convention, and then looked in detail,
what are the rectangles to be plotted. Let us consider the sample sets
as below. Recall these were the same sample sets we saw in the beginning
of this section. Note the TIA is already calculated indicating us the
kind of relationship.
\begin{figure}
\centering% remove input part of cells with tag to_remove
    %((- if cell.metadata.hide_input -))
    \begin{center}
    \adjustimage{max size={0.9\linewidth}{0.9\paperheight},min size={0.5\linewidth}{!}}{30_Covariance_3_files/30_Covariance_3_4_0.pdf}
    \end{center}
    { \hspace*{\fill} \\}
    \caption{Figure 1: The Sample Sets} \label{fig:C3_002}
\end{figure}
    As per TIA from Figure \(\ref{fig:C3_002}\), Plot 1 is highly negatively
correlated, Plot 2 is somewhat negative, and Plot 3 is positively
correlated. Though visually Plots 1 and 3 look like not having much
difference in their \emph{slope} or \emph{rate}, our TIA gives a wide
difference in value. This is because, TIA between sample sets are not
comparable (we will solve that soon in correlation, but remember this
problem). That is, given a sample set, say Plot 1, having -148859 is one
of infinite no of possibilities among that sample set, with perfectly
linear positive, negative and 0 TIA as one of those. Simiarly for sample
set in Plot 2 and so on. Below are the sample sample plots with colored
rectangles laid over them. Remember, if N is the size of sample set, or
no of \((x,y)\) pairs, then the number of rectangles we have drawn is
\(N(N-1)/2\). And as we already saw, only because of this limited
rectangles, we get the output as below without neutralization issues.

\begin{figure}
\centering% remove input part of cells with tag to_remove
    %((- if cell.metadata.hide_input -))
    \begin{center}
    \adjustimage{max size={0.9\linewidth}{0.9\paperheight},min size={0.5\linewidth}{!}}{30_Covariance_3_files/30_Covariance_3_8_0.pdf}
    \end{center}
    { \hspace*{\fill} \\}
    \caption{Figure 2: The Visualization of Covariance} \label{fig:C3_002}
\end{figure}
    I think, Figure \(\ref{fig:C3_002}\) speaks for itself :) Plot 1,which
has highly negative linear relationship among its sample sets, has more
red rectangles than green. Plot 2, which is very less linearity in any
direction, shows an almost equal mix of red and green, of course the
accurate measure is reflected in its TIA though. Plot 3, which has a
positive linear relationship, obviously has lot more green. Figure
\(\ref{fig:C3_003}\) gives total area of red and green separately,
giving us better glimpse of the \emph{net} relationship underneath. The
TIA is just the difference between the total green area and red area.

\begin{figure}
\centering% remove input part of cells with tag to_remove
    %((- if cell.metadata.hide_input -))
    \begin{center}
    \adjustimage{max size={0.9\linewidth}{0.9\paperheight},min size={0.5\linewidth}{!}}{30_Covariance_3_files/30_Covariance_3_12_0.pdf}
    \end{center}
    { \hspace*{\fill} \\}
    \caption{Figure 3: The separated total area \\Green indicates Positive} \label{fig:C3_003}
\end{figure}
    \section{Expected value of TIA}\label{expected-value-of-tia}

For any given sample set, we are typically interested not in the total
of the sample set, but most probable or best representative candidate of
that sample set. In our case, our sample set of TIA, is not individual
pairs \((x_i,y_i)\), but a function of them, a product
\((x_i-x_j)(y_i-y_j)\). That is, using \(\ref{eq:C002}\) if,\\
\[
h(X,Y) = \sum\limits_{i=1}^N\sum\limits_{j=i+1}^{N}(x_i - x_j)(y_i - y_j)
\]

then, we are interested in \(E[h(X,Y)]\)

As per expectation formula,

\begin{equation}
E[h(X,Y)] =  \sum\limits_{i=1}^N\sum\limits_{j=i+1}^{N}(x_i - x_j)(y_i - y_j)p(x_i , y_i) \label{eq:C3_001}
\end{equation}

\begin{quote}
Note, we are not interested in expected value of \emph{number of
rectangles} or \emph{red colored rectangles} etc. The area of rectangles
carry the measure and each rectangle might have different area. We are
thus interested in the \emph{expected value} of the area, given the
\emph{total} interested area.
\end{quote}

\emph{Expectation} needs a \emph{joint probability mass function}
\(p(X,Y)\) associated with \(h(X,Y)\). Recall the rectangle graph for
\(N=6\) and replace with area \(A_{ij}\) (could also call as product,
\(P_{ij}\) but just to avoid notational confusion with probability let
us stick with area).
% remove input part of cells with tag to_remove
    %((- if cell.metadata.hide_input -))
    ~

\begin{figure}
\centering% remove input part of cells with tag to_remove
    %((- if cell.metadata.hide_input -))
    \begin{center}
    \adjustimage{max size={0.9\linewidth}{0.9\paperheight},min size={0.5\linewidth}{!}}{30_Covariance_3_files/30_Covariance_3_18_0.pdf}
    \end{center}
    { \hspace*{\fill} \\}
    \caption{Figure 4: The Number of Area Components} \label{fig:C3_004}
\end{figure}
    Assuming each \emph{area} has equal probability, given the number of
area, each \(A_{ij}\) will have a probability of \(\dfrac{1}{N^2}\) as
there are \(N^2\) area components possible. Thus, \(\ref{eq:C3_001}\)
becomes,

    \begin{equation}
E[h(X,Y)] =  \sum\limits_{i=1}^N\sum\limits_{j=i+1}^{N}(x_i - x_j)(y_i - y_j)p(x_i , y_i) =  \dfrac{1}{N^2}\sum\limits_{i=1}^N\sum\limits_{j=i+1}^{N}(x_i - x_j)(y_i - y_j) \label{eq:C3_002}
\end{equation}

    Ladies and Gentlemen. That \(E[h(X,Y)]\) is called \textbf{Covariance}
of X and Y , shortly called \(\mathbf{\mathrm{Cov}(X,Y)}\). Also note,
the alternative form we saw earlier in equation \(\ref{eq:C003}\), could
also be used to derive covariance as below.

    \begin{equation}
E[h(X,Y)] =  \sum\limits_{i=1}^N\sum\limits_{j=i+1}^{N}(x_i - x_j)(y_i - y_j)p(x_i , y_i) =  \dfrac{1}{2N^2}\sum\limits_{i=1}^N\sum\limits_{j=1}^{N}(x_i - x_j)(y_i - y_j) \label{eq:C3_003}
\end{equation}

\begin{tcolorbox}[colback=green!5,colframe=green!40!black,title=Covariance of discrete X and Y with p(X\,Y) uniform]
Given X and Y are discrete variables of sample size N, and $p(X,Y) = \dfrac{1}{N^2}$, 
\begin{equation}
    \mathrm{Cov}(X,Y) = \dfrac{1}{N^2}\sum\limits_{i=1}^N\sum\limits_{j=i+1}^{N}(x_i - x_j)(y_i - y_j) \label{eq:C3_004}
\end{equation}
\begin{equation}
    \mathrm{Cov}(X,Y) = \dfrac{1}{2N^2}\sum\limits_{i=1}^N\sum\limits_{j=1}^{N}(x_i - x_j)(y_i - y_j) \label{eq:C3_005}
\end{equation}
\end{tcolorbox}
    \section{Standard Formula}\label{standard-formula}

What we have seen so far, is a deformed form of covariance which
numerically gave us the same results as a standard formula. It is
mathematically possible to show that,

\begin{equation}
\mathrm{Cov}(X,Y) = \sum\limits_{i=1}^N\sum\limits_{j=i+1}^{N}(x_i - x_j)(y_i - y_j)p(x_i , y_i) = \sum\limits_{i=1}^{N}(x_i - \overline{x})(y_i - \overline{y}) p(x_i , y_i)
\end{equation}

The derivation is proven by \citet{yuli2012}. 
    At the time of this writing, the
\href{https://math.stackexchange.com/questions/2982674/single-to-double-summation-via-vector-matrix-problem}{doubts}
in the derivation is not yet cleared, if and once it is done, this
section should be enriched with a proper derivation. Till then, this is
a discontinuity in our understanding. The visualization of standard
formula is slightly different because it involves mean, so all
rectangles have one corner at mean position
\((\overline{x},\overline{y})\). The visualization is shown in figure
\(\ref{fig:C3_005}\). The top 3 rows from our deformed formula and
bottom 3 using standard formula. One could observe, the rectangles in
plots 3,4,and 5 are centered around the mean (shown in dotted lines),
thus giving a better viusal perception of the measure (no of red or
green rectangles, which is more). We did not start with this
visualization only because, there was no intuition to introduce mean in
the equation out of no where.

\begin{figure}
\centering% remove input part of cells with tag to_remove
    %((- if cell.metadata.hide_input -))
    \begin{center}
    \adjustimage{max size={0.9\linewidth}{0.9\paperheight},min size={0.5\linewidth}{!}}{30_Covariance_3_files/30_Covariance_3_29_0.pdf}
    \end{center}
    { \hspace*{\fill} \\}
    \caption{Figure 5: The Visualization of deformed and standard formula\\for Covariance} \label{fig:C3_005}
\end{figure}


    % Add a bibliography block to the postdoc
    
    
    
    \end{document}
